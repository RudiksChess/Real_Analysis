\section{21 de enero de 2021}
\begin{example}
$d_2: \mathbb{R}^n\times \mathbb{R}^n \to d_2(x,y)=\sqrt{\suma{i=1}{n}{(x_i-y_i)^"}}$ es métrica. A probar $$d_2(x,y)\leq d_2(x,z)+d_2(z,y)$$
\end{example}

\begin{proof}

\begin{align}
    [d_2(x,z)+d_2(z,y)]^2 = \\
    [\sqrt{\sum(x_i-z_i)^2}
    +
    \sqrt{\sum(z_i-y_i)^2}]\\
    = \sum(x_i-z_i)^2+\sum(z_i-y_i)^2+2[(\sum(x_i-z_i)^2)(\sum (z_i-y_i)^2]^{1/2}\geq\\ \sum(x_i-z_i)^2+\sum(z_i-y_i)^2+2\sum (x_i-z_i)(z_i-y_i)\\
    \marginnote{Cauchy-Schwarz $[(\sum a_ib_i)^2\leq (\sum a_i^2)(\sum b_i^2)]^{1/2}$}
    = \sum [(x_i-z_i)^2+2(x_i-z_i)(z_i-y_i)+(z_i-y_i)^2]\\
    = \sum[(x_i-z_i)+(z_i-y_i)]^2\\
    \to d_2(x,z)+d_2(z,y)\geq d_2(x,y)
\end{align}
\end{proof}

\begin{example}
Considere: $\mathbb{R}^n\times \mathbb{R}^n \to \mathbb{R}\ni$ $$d_\infty= máx\s{|x_i-y_i|: i=1,...,n}$$
$\to d_\infty$ es una métrica de $\mathbb{R}^n$. 
\marginnote{Tenemos $x=(2,3,4)$ y $y=(-1,2,0)$, entonces $d_\infty(x,y)=máx\s{|2-(-1)|,|3-2|,|4-0|}= máx\s{3,1,4}=4$}
\end{example}

\begin{example}
Sea $B([a,b])$ el conjunto de funciones acotadas definidas en $[a,b]$ y de valores reales. También se denota: 
$$l^\infty([a,b])=\s{f:[a,b]\to \mathbb{R}\ni |f(x)|\leq M, M>0}$$

\textbf{Ejemplos}\newline 
Dadas $f,g\in l^{\infty}[a,b]$

\marginnote{$$f(x)=\begin{cases}
1, x\in \mathbb{Q}\\
0, x\in \mathbb{I}
\end{cases}$$
Función de Dirichlet }

\begin{center}
    

\tikzset{every picture/.style={line width=0.75pt}} %set default line width to 0.75pt        

\begin{tikzpicture}[x=0.75pt,y=0.75pt,yscale=-1,xscale=1]
%uncomment if require: \path (0,300); %set diagram left start at 0, and has height of 300

% Plotting does not support converting to Tikz
%Straight Lines [id:da13236771746301135] 
\draw [color={rgb, 255:red, 255; green, 50; blue, 0 }  ,draw opacity=1 ]   (145.8,113) -- (144.8,240) ;
%Straight Lines [id:da30162929751267176] 
\draw [color={rgb, 255:red, 255; green, 50; blue, 0 }  ,draw opacity=1 ]   (518.8,131) -- (516.8,239) ;
%Straight Lines [id:da9538754545706293] 
\draw [color={rgb, 255:red, 255; green, 0; blue, 4 }  ,draw opacity=1 ] [dash pattern={on 4.5pt off 4.5pt}]  (126,106) -- (517.8,106) ;
%Straight Lines [id:da8164736652922968] 
\draw [color={rgb, 255:red, 255; green, 0; blue, 4 }  ,draw opacity=1 ] [dash pattern={on 4.5pt off 4.5pt}]  (124,194) -- (515.8,194) ;
%Straight Lines [id:da18688557822586516] 
\draw [color={rgb, 255:red, 248; green, 231; blue, 28 }  ,draw opacity=1 ]   (210.8,114) -- (209.8,241) ;
%Straight Lines [id:da8090779729750256] 
\draw    (311.96,108) -- (310.84,190) ;
\draw [shift={(310.8,193)}, rotate = 270.78] [fill={rgb, 255:red, 0; green, 0; blue, 0 }  ][line width=0.08]  [draw opacity=0] (8.93,-4.29) -- (0,0) -- (8.93,4.29) -- cycle    ;
\draw [shift={(312,105)}, rotate = 90.78] [fill={rgb, 255:red, 0; green, 0; blue, 0 }  ][line width=0.08]  [draw opacity=0] (8.93,-4.29) -- (0,0) -- (8.93,4.29) -- cycle    ;
%Straight Lines [id:da6781602044348318] 
\draw    (361.8,84) -- (326.34,123.51) ;
\draw [shift={(325,125)}, rotate = 311.90999999999997] [color={rgb, 255:red, 0; green, 0; blue, 0 }  ][line width=0.75]    (10.93,-3.29) .. controls (6.95,-1.4) and (3.31,-0.3) .. (0,0) .. controls (3.31,0.3) and (6.95,1.4) .. (10.93,3.29)   ;

% Text Node
\draw (103,222.1) node [anchor=north west][inner sep=0.75pt]    {$0$};
% Text Node
\draw (288,224.1) node [anchor=north west][inner sep=0.75pt]    {$x$};
% Text Node
\draw (139,242.1) node [anchor=north west][inner sep=0.75pt]    {$a$};
% Text Node
\draw (510,254.1) node [anchor=north west][inner sep=0.75pt]    {$b$};
% Text Node
\draw (211,255.1) node [anchor=north west][inner sep=0.75pt]    {$x$};
% Text Node
\draw (353,57.1) node [anchor=north west][inner sep=0.75pt]    {$d_{_{\infty }}( f,g)$};


\end{tikzpicture}
\end{center}

$$\to d_\infty(f,g)=sup_{x\in[a,b]}\s{|f(x)-g(x)|}$$, la cual es una métrica en $l^{\infty}[a,b]$ y se llama métrica o distancia del supremo.
\end{example}

\begin{example}
Sea $C[a,b]$ el conjunto de funciones continuas sobre el $[a,b]$ con valores reales. Entonces, si $f,g\in C[a,b]$, se tiene: la métrica $d(f,g)=\int_a^b |f(x)-g(x)|dx$ sobre $C[a,b]$
\end{example}

\begin{definition}[Norma]
Suponga que $V$ es un espacio vectorial sobre el campo $\mathbb{F}(\mathbb{R} o \mathbb{C})$ y que $$\norm{\cdot}: V\to \mathbb{R}\ni$$
$\forall x,y\in V$ y $\alpha\in \mathbb{F}$, se cumplen:
\begin{itemize}
    \item $\norm{x}\geq 0, \norm{x}=0$ ssi $x=0$
    \item $\norm{\alpha x}=|\alpha|\cdot \norm{x}$
    \item $\norm{x+y}\leq \norm{x}+\norm{y}$
    
    Entonces, $\norm{\cdot}$ es una norma sobre $V$, y decimos que $(V,\norm{\cdot})$ es un espacio normado.
\end{itemize}
\end{definition}

\begin{remark}
Sea $V$ un espacio vectorial normado. Entonces, considere: 
$$d:V\times V\to \mathbb{R}\ni$$
$$d(x,y)=\norm{x-y}$$
Nótese que: \begin{itemize}
    \item $d(x,y)=\norm{x.y}\geq 0;$\newline 
    Si $x=y\to d(x,y)=\norm{x-y}=0$
    \newline 
    Si $d(x,y)=\norm{x-y}=0\to x-y=0\to x=y$
    \item $d(x,y)=\norm{x-y}=\norm{-(y-x)}=|-1|\norm{y-x}= \norm{y-x}=d(y,x)$
    \item $d(x,y)=\norm{x-y}=\norm{(x-z)+(z-y)}\leq \norm{x-z}+\norm{a}$
    \newline $d(x,y)=\norm{x-y}$ es una métrica sobre $V$. Esta es la métrica inducida por la norma. 
\end{itemize}
\end{remark}


\subsection{Topología de $\mathbb{R}$ ($\mathbb{R}^n$)}
\begin{definition}
Sea $x$ un conjunto no vacío. Una familia de subconjunto $\tau$ de $x$ es una topología sobre $x$ si: 
\begin{itemize}
    \item $\emptyset, x\in \tau$
    \item Cualquier familia $\s{A_i}$ de elementos de $\tau$ es t.q. $\bigcup_i A_i\in \tau$
    \item Si $A_1,A_2\in \tau\to A_1\cap A_2\in \tau$. A los elementos de $\tau$ se les llama abiertos de $x$. 
\end{itemize}
\end{definition}