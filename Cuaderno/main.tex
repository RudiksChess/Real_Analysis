\documentclass[
	fontsize=10pt, % Base font size
	twoside=false, % Use different layouts for even and odd pages (in particular, if twoside=true, the margin column will be always on the outside)
	%open=any, % If twoside=true, uncomment this to force new chapters to start on any page, not only on right (odd) pages
	%chapterprefix=true, % Uncomment to use the word "Chapter" before chapter numbers everywhere they appear
	%chapterentrydots=true, % Uncomment to output dots from the chapter name to the page number in the table of contents
	numbers=noenddot, % Comment to output dots after chapter numbers; the most common values for this option are: enddot, noenddot and auto (see the KOMAScript documentation for an in-depth explanation)
	%draft=true, % If uncommented, rulers will be added in the header and footer
	%overfullrule=true, % If uncommented, overly long lines will be marked by a black box; useful for correcting spacing problems
]{kaobook}

% Set the language
\usepackage[spanish]{babel} % Load characters and hyphenation
\usepackage[english=british]{csquotes} % English quotes

% Load packages for testing
\usepackage{blindtext}
%\usepackage{showframe} % Uncomment to show boxes around the text area, margin, header and footer
%\usepackage{showlabels} % Uncomment to output the content of \label commands to the document where they are used

% Load the bibliography package
\usepackage{styles/kaobiblio}
\addbibresource{main.bib} % Bibliography file

% Load mathematical packages for theorems and related environments. NOTE: choose only one between 'mdftheorems' and 'plaintheorems'.
\usepackage{styles/mdftheorems}
%\usepackage{styles/plaintheorems}
\usepackage{etoolbox}
\AtBeginEnvironment{align}{\setcounter{equation}{0}}

\newcommand{\al}{\alpha}
\newcommand{\be}{\beta}
\newcommand{\g}{\gamma}
\newcommand{\p}{\varphi}
\newcommand{\up}{\upsilon}
\newcommand{\s}[1]{\{#1\}}
\newcommand{\la}[1]{\mathcal{L}[#1]}
\newcommand{\lai}[1]{\mathcal{L}^{-1}[#1]}
\newcommand{\suma}[3]{\sum_{#1}^{#2}{#3}}
\newcommand{\mi}[1]{&& #1 &&}
\newcommand{\bv}[1]{\langle #1 \rangle}
\newcommand{\uvec}[1]{\boldsymbol{\hat{\textbf{#1}}}}
\newcommand{\vf}[3]{#1\uvec{i}#2\uvec{j}#3\uvec{k}}
\newcommand{\vfd}[2]{#1\uvec{i}#2\uvec{j}}
\newcommand{\ej}[1]{\begin{exercise}
\begin{align}
    #1
\end{align}
\end{exercise}
}
\newcommand{\dn}[1]{\begin{definition}
\begin{align}
    #1
\end{align}
\end{definition}
}
\newcommand{\ta}[1]{\begin{theorem}
\begin{align}
    #1
\end{align}
\end{theorem}
}
\newcommand{\derivada}[2]{\frac{d#1}{d#2}}

\newcommand{\ieo}[4]{\oint_{#1}^{#2} #3 \, d{#4}}
\newcommand{\ied}[4]{\int_{#1}^{#2}\int_{#3}^{#4}}
\newcommand{\norm}[1]{\left\lVert#1\right\rVert}
\newcommand{\rmx}[1]{\begin{pmatrix}#1\end{pmatrix}}




\graphicspath{{examples/documentation/images/}{images/}} % Paths in which to look for images

\makeindex[columns=3, title=Alphabetical Index, intoc] % Make LaTeX produce the files required to compile the index

\makeglossaries % Make LaTeX produce the files required to compile the glossary

\makenomenclature % Make LaTeX produce the files required to compile the nomenclature

% Reset sidenote counter at chapters
%\counterwithin*{sidenote}{chapter}

%----------------------------------------------------------------------------------------

\begin{document}

%----------------------------------------------------------------------------------------
%	BOOK INFORMATION
%----------------------------------------------------------------------------------------

\titlehead{Primer semestre 2021 - Matemática Aplicada - Universidad del Valle de Guatemala}
\subject{Un nuevo mundo...}

\title[Análisis de Variable Real 1]{Análisis de Variable Real 1}


\author[Rudik Roberto Rompich]{Rudik Roberto Rompich}

\date{Enero a Junio de 2021}

%----------------------------------------------------------------------------------------

\frontmatter % Denotes the start of the pre-document content, uses roman numerals

%----------------------------------------------------------------------------------------
%	OPENING PAGE
%----------------------------------------------------------------------------------------

%\makeatletter
%\extratitle{
%	% In the title page, the title is vspaced by 9.5\baselineskip
%	\vspace*{9\baselineskip}
%	\vspace*{\parskip}
%	\begin{center}
%		% In the title page, \huge is set after the komafont for title
%		\usekomafont{title}\huge\@title
%	\end{center}
%}
%\makeatother

%----------------------------------------------------------------------------------------
%	COPYRIGHT PAGE
%----------------------------------------------------------------------------------------

\makeatletter
\uppertitleback{\@titlehead} % Header

\lowertitleback{
	\textbf{Disclaimer}\\
	Supuestamente, una parte elemental de la matemática.
	\medskip
	
	\textbf{No copyright}\\
	\cczero\ This book is released into the public domain using the CC0 code. To the extent possible under law, I waive all copyright and related or neighbouring rights to this work.
	
	To view a copy of the CC0 code, visit: \\\url{http://creativecommons.org/publicdomain/zero/1.0/}
	
	\medskip
	
	\textbf{Colophon} \\
	This document was typeset with the help of \href{https://sourceforge.net/projects/koma-script/}{\KOMAScript} and \href{https://www.latex-project.org/}{\LaTeX} using the \href{https://github.com/fmarotta/kaobook/}{kaobook} class.
	
	The source code of this book is available at:\\\url{https://github.com/fmarotta/kaobook}
	
	(You are welcome to contribute!)
	
	\medskip
	
	\textbf{Publisher} \\
	Rudik Roberto Rompich Cotzojay
}
\makeatother

%----------------------------------------------------------------------------------------
%	DEDICATION
%----------------------------------------------------------------------------------------

\dedication{
	The harmony of the world is made manifest in Form and Number, and the heart and soul and all the poetry of Natural Philosophy are embodied in the concept of mathematical beauty.\\
	\flushright -- D'Arcy Wentworth Thompson
}

%----------------------------------------------------------------------------------------
%	OUTPUT TITLE PAGE AND PREVIOUS
%----------------------------------------------------------------------------------------

% Note that \maketitle outputs the pages before here

% If twoside=false, \uppertitleback and \lowertitleback are not printed
% To overcome this issue, we set twoside=semi just before printing the title pages, and set it back to false just after the title pages
\KOMAoptions{twoside=semi}
\maketitle
\KOMAoptions{twoside=false}

%----------------------------------------------------------------------------------------
%	PREFACE
%----------------------------------------------------------------------------------------

%\input{chapters/preface.tex}

%----------------------------------------------------------------------------------------
%	TABLE OF CONTENTS & LIST OF FIGURES/TABLES
%----------------------------------------------------------------------------------------

\begingroup % Local scope for the following commands

% Define the style for the TOC, LOF, and LOT
%\setstretch{1} % Uncomment to modify line spacing in the ToC
%\hypersetup{linkcolor=blue} % Uncomment to set the colour of links in the ToC
\setlength{\textheight}{23cm} % Manually adjust the height of the ToC pages

% Turn on compatibility mode for the etoc package
\etocstandarddisplaystyle % "toc display" as if etoc was not loaded
\etocstandardlines % toc lines as if etoc was not loaded

\tableofcontents % Output the table of contents

\listoffigures % Output the list of figures

% Comment both of the following lines to have the LOF and the LOT on different pages
\let\cleardoublepage\bigskip
\let\clearpage\bigskip

\listoftables % Output the list of tables

\endgroup

%----------------------------------------------------------------------------------------
%	MAIN BODY
%----------------------------------------------------------------------------------------

\mainmatter % Denotes the start of the main document content, resets page numbering and uses arabic numbers
\setchapterstyle{kao} % Choose the default chapter heading style

\setchapterpreamble[u]{\margintoc}
\chapter{Análisis de Variable Real}
\labch{intro}

\section{11 de enero de 2021}
\subsection{Propiedades de R}

\begin{definition}
Un conjunto no vacío $P$ de elementos de un campo $\mathbb{F}$ es una clase positiva si cumple: 
\begin{itemize}
    \item Si $a,b\in P\to a+b \in P$
    \item Si $a,b \in P\to a\cdot b \in P$
    \item Si $a\in\mathbb{F}$, entonces:\newline
    \textbf{Ley de tricotomía}\newline
    $a\in P\quad\quad  O_{ex} \quad\quad  a=0 \quad\quad  O_{ex} \quad\quad  -a\in P$ 
\end{itemize}
\end{definition}
\begin{remark}
Sea $N=\s{-a: a\in P}$ la clase negativa relativa a $P$. $\to \mathbb{F}= P\cup \s{0}\cup N$
\end{remark}

\begin{example}
Algunos ejemplos...
\begin{itemize}
    \item $(Q,+,\cdot)$ es un campo. Sea $P=\s{\frac{a}{b}\in Q\ni a,b \in \mathbb{Z}^+}\to P$ es una clase positiva de $Q$.
    \item Sea $\mathbb{Z}_2 =\s{0,1}$, con las operaciones: 
    

\tikzset{every picture/.style={line width=0.75pt}} %set default line width to 0.75pt        

\begin{tikzpicture}[x=0.75pt,y=0.75pt,yscale=-1,xscale=1]
%uncomment if require: \path (0,350); %set diagram left start at 0, and has height of 350

%Shape: Boxed Line [id:dp6291595595512527] 
\draw    (288.6,132) -- (197.6,131) ;
%Straight Lines [id:da0037758895887590738] 
\draw    (230.6,102) -- (229.6,193) ;
%Straight Lines [id:da8491580753688753] 
\draw    (260.6,102) -- (259.6,193) ;
%Shape: Boxed Line [id:dp07775959649029429] 
\draw    (289.6,161) -- (198.6,160) ;
%Shape: Boxed Line [id:dp8480833689358503] 
\draw    (425.6,131) -- (334.6,130) ;
%Straight Lines [id:da6845007899647696] 
\draw    (367.6,101) -- (366.6,192) ;
%Straight Lines [id:da890039541275841] 
\draw    (397.6,101) -- (396.6,192) ;
%Shape: Boxed Line [id:dp9051592988280287] 
\draw    (426.6,160) -- (335.6,159) ;

% Text Node
\draw (346,104.1) node [anchor=north west][inner sep=0.75pt]    {$*$};
% Text Node
\draw (205,105.1) node [anchor=north west][inner sep=0.75pt]    {$+$};
% Text Node
\draw (239,105.9) node [anchor=north west][inner sep=0.75pt]   [align=left] {0};
% Text Node
\draw (208,135.9) node [anchor=north west][inner sep=0.75pt]   [align=left] {0};
% Text Node
\draw (345,136.9) node [anchor=north west][inner sep=0.75pt]   [align=left] {0};
% Text Node
\draw (375,108.9) node [anchor=north west][inner sep=0.75pt]   [align=left] {0};
% Text Node
\draw (207,167.9) node [anchor=north west][inner sep=0.75pt]   [align=left] {1};
% Text Node
\draw (271,105.9) node [anchor=north west][inner sep=0.75pt]   [align=left] {1};
% Text Node
\draw (345,167.9) node [anchor=north west][inner sep=0.75pt]   [align=left] {1};
% Text Node
\draw (409,105.9) node [anchor=north west][inner sep=0.75pt]   [align=left] {1};
% Text Node
\draw (270,137.9) node [anchor=north west][inner sep=0.75pt]   [align=left] {1};
% Text Node
\draw (238,167.9) node [anchor=north west][inner sep=0.75pt]   [align=left] {1};
% Text Node
\draw (406,167.9) node [anchor=north west][inner sep=0.75pt]   [align=left] {1};
% Text Node
\draw (239,138.9) node [anchor=north west][inner sep=0.75pt]   [align=left] {0};
% Text Node
\draw (268,167.9) node [anchor=north west][inner sep=0.75pt]   [align=left] {0};
% Text Node
\draw (375,135.9) node [anchor=north west][inner sep=0.75pt]   [align=left] {0};
% Text Node
\draw (405,137.9) node [anchor=north west][inner sep=0.75pt]   [align=left] {0};
% Text Node
\draw (377,165.9) node [anchor=north west][inner sep=0.75pt]   [align=left] {0};
\end{tikzpicture}
\newline
$\to (\mathbb{Z}_2,+,\cdot)$ es un campo. 
\item Sea $P=\s{0}$. \newline (1) Cumple (2) Cumple (3) No cumple $\to P$ no es una clase positiva de $\mathbb{Z}_2$.
\item Sea $P'=\s{1}$\newline (1) No cumple. $\to P'$ no es una clase positiva de $\mathbb{Z}_2$.
\end{itemize}
\end{example}

\begin{definition}
Sea $P$ la clase positiva del campo $\mathbb{F}$, entonces se dice que $\mathbb{F}$ está ordenada por $P$. (O que $\mathbb{F}$ es un campo ordenado).
\begin{itemize}
    \item Si $a\in P$, se dice que $a$ es positivo. Notación $a>0$. 
    \item Si $a\in P$ o $a=0$, se dice que $a$ es no negativo. Notación $a\geq 0$
    \item Si $a,b \in \mathbb{F}$ y $a-b\in P$, se escribe $a>b$.
    \item Si $a,b\in \mathbb{F}$ y $a-b\in P$ o $a-b=0$, se escribe $a\geq b$. 
\end{itemize}
\end{definition}

\begin{proposition}
Algunas propiedades...
\begin{itemize}
    \item{Transitividad} Si $a>b$ y $b>c\to a>c$.
    \item{Tricotomía} Si $a,b\in \mathbb{F}$, entonces: \newline $a>b\quad\quad O_{ex}\quad\quad a=b \quad \quad O_{ex}\quad \quad b>a$.
    \item{Antisimetría} Si $a\geq b$ y $b\geq a \to a=b$
\end{itemize}
\end{proposition}
\begin{proof}
Se presentan las pruebas: 
\begin{itemize}
    \item {1.} Si $$a>b\to a-b\in P$$
                  $$b>c\to b-c \in P$$
                  $$\to (a-b)+(b-c)= a-c \in P$$
                  $$\to a>c$$
    \item {2.} Por la tricotomía en $P$.
    \item {3.} Supóngase, por el absurdo, que $a\geq b$, $b\geq a$ y $a\neq b$. Entonces $a>b\quad \quad O_{ex} \quad \quad b>a (\to\gets)$
\end{itemize}
\end{proof}
\begin{proposition}
Sea $\mathbb{F}$ un campo ordenado: 
\begin{itemize}
    \item Si $a\neq 0\to a^2>0$
    \item 1>0
    \item Si $n\in \mathbb{Z}^+\to n>0$
\end{itemize}
\end{proposition}

\begin{proof}
Pruebas\\
\textbf{1.}
\begin{itemize}
\item Si $a\neq 0 \to a>0\quad\quad O_{ex}\quad \quad a<0$.
\item Si $a>0\to a\cdot a= a^2 >0$\\ (Por la propiedad de $a\in P \to a\cdot a \in P \to a^2>0$).
\item Si $a<0 \to -a\in P\to (-a)\cdot (-a) = a^2\in P\to a^2>0$.
\end{itemize}
\textbf{2.}
\begin{itemize}
    \item Hacemos $a=1$ en 1) $\to 1^2=1>0$. 
\end{itemize}
\textbf{3.} Por inducción sobre $n$.
\begin{itemize}
    \item Probemos para $n=1$. Por 2), $1>0$.
    \item Suponemos que $k>0$.
    \item Probamos para $k+1$; $k>0,1>0 \to k+1>0\to n>0, \forall n\in \mathbb{Z}^+$
\end{itemize}
\end{proof}

\begin{theorem}
Sean $a,b,c\in \mathbb{F}$...
\begin{itemize}
    \item  Si $a>b\to a+c>b+c$
    \item Si $a>b$ y $c>d\to a+c>b+d$
    \item Si $a>b$ y $c>0\to a\cdot c >b\cdot c$
    \item Si $a>b$ y $c<0\to a\cdot c<b\cdot c$
    \item Si $a>0\to a^{-1}>0$
    \item Si $a<0\to a^{-1}<0$
\end{itemize}
\end{theorem}
\begin{proof}
Prueba...
\begin{enumerate}
    \item Si $a>b\to a-b\in P\leftrightarrow (a+c)-(b+c)\in P\to a+c >b+c$.
    \item Si $$a>b\to a-b\in P$$
    $$c>d\to c-d \in P $$ $$\to (a-b)+(c-d)=(a+c)-(b+c)\in P$$
    $$\to a+c >b+c$$
    \item Si $$a>b\to a-b\in P$$
    $$c>0 \to c\in P$$
    $$\to (a-b)\cdot c = ac-bc\in P\to ac>bc $$
    \item Si $$a>b\to a-b\in P$$
            $$c<0 \to -c\in P$$
            $$\to (a-b)\cdot (-c)= bc-ac\in P$$
            $$\to bc>ac$$
    \item Sea $a>0$ y suponemos $a^{-1}<0\to -a^{-1}>0 \to a(-a^{-1})>0$, pero $a(-a^{-1})=-1>0(\to\gets)$. Si $a^{-1}=0\to 1=aa^{-1}=0 (\to\gets)\to a^{-1}>0$. 
    \item Similar al 5) 
\end{enumerate}
\end{proof}

\begin{corollary}
Si $a>b\to a>\frac{a+b}{2}>b$
\end{corollary}

\begin{proof}
\begin{align}
    \mi{a>b &\to a+a>a+b}\\
    \mi{&\to 2a>a+b}\\
    \mi{&\to a>\frac{a+b}{2}}
\intertext{Por otro lado}
    \mi{a>b &\to a+b>b+b}\\
    \mi{&\to a+b>2b}\\
    \mi{&\to \frac{a+b}{2}>b}\\
\mi{&\therefore a>\frac{a+b}{2}>b}
\end{align}
\marginnote[-12   pt ]{En el corolario, hagamos $b=0$. Entonces, si $a>0\implies a>\frac{a}{2}>0$. Entonces, en un campo ordenado no existe un número positivo menor.}
\end{proof}

\begin{theorem}
Si $ab>0$, entonces, $a>0$ y $b>0$ o $a<0$ y $b<0$.
\end{theorem} 

\begin{proof}
Pendiente
\end{proof}

\subsection{Valor absoluto}
\begin{definition}
Sea $\mathbb{F}$ un campo con clase positiva $P$. Se define la función valor absoluto: 
$$|\cdot|: \mathbb{F}\to P \cup \s{0}\ni $$
$$|a|=\begin{cases} 
      a & \text{si } a\geq 0 \\
      -a & \text{si } a<0
   \end{cases}$$
\end{definition}

\begin{theorem}
Related
\begin{enumerate}
    \item $|a|=0$ ssi $a=0$
    \item $|-a|=|a|$
    \item $|ab|=|a|\cdot|b|$
    \item Si $c\geq 0\to |a|\leq c$ ssi $-c\leq a\leq c$
\end{enumerate}
\end{theorem}

\begin{proof}
Pruebas...
\begin{enumerate}
    \item 
    ($\to$) Si $a=0\to$ Por definición: $|a|=|0|=0$.
    \newline 
    ($\gets$) A probar: Si $|a|=0\to a=0$
    \newline 
    \textit{Por contrapuesta,} suponga que $a\neq 0\to a>0$ o $a<0$.  
    \newline 
    Si $a>0\to |a|=a>0\to |a|\neq 0$
    \newline 
    Si $a<0\to |a|=-a>0= |a|\neq 0$
    \newline 
    i.e. si $a\neq 0\to |a|= 0$
    \item A probar: $-|a|=|a|$
    \newline
    Si $a>0\to |a|=a=|-a|$
    \newline 
    Si $a<0\to |a|=-a=|-a|$
    \newline 
    Si $a=0\to |0|=0=|-0|$
    
    \item $|ab|=|a|\cdot |b|$\newline 
    Por casos:
    \begin{itemize}
        \item Si $a>0$ y $b>0$, entonces: $$ab>0\to |ab|=ab=|a|\cdot |b|$$
        \item Si $a>0$ y $b<0\to ab<0\to |ab|=-(ab)=a\cdot (-b)=|a|\cdot |b|$
        \item Si $a<0$ y $b<0\to ab>0\to |ab|=ab=(-a)(-b)=|a|\cdot |b|$
    \end{itemize}
    \item A probar: si $c\geq 0\to |a|\leq c$ ssi $-c\leq a\leq c$
    \marginnote[-12   pt ]{$a\leq b\qquad -a\leq b \leftrightarrow |a|\leq b$}
    \newline 
    ($\to$) Suponga que $|a|\leq c\to a\leq c$ y $-a\leq c \to a\leq c$ y $a\geq -c, \therefore -c\leq a\leq c$
    \newline 
    ($\gets$) Suponga que $-c\leq a\leq c \to -c\leq a $ y $a\leq c\to c\geq -a$ y $a\leq c$ $\therefore |a|\leq c$
\end{enumerate}
\end{proof}



\section{14 de enero de 2021}
\begin{remark}
Como el valor absoluto de $|a|\geq 0\to -|a|\leq |a|\to -|a|\leq a\leq|a|, \forall a $
\end{remark}

\begin{theorem}[Desigualdad triangular]
Sea $a$ y $b$ elementos de un campo ordenado $\mathbb{F}$. Entonces $|a+b|\leq |a|+|b|$
\end{theorem}
\begin{proof}
\begin{align}
    -|a|\leq & a \leq |a|\\
    -|b|\leq & b \leq |b|\\
    -(|a|+|b|)\leq & a+b \leq |a|+|b|\\
    |a+b|\leq & |a|+|b|
\end{align}
\end{proof}

\begin{remark}
Si $a,b$ son elementos del campo ordenado $F$, entonces: 
\begin{itemize}
    \item 
    \begin{align}
    ||a|-|b|| & \leq |a-b|
    \intertext{En efecto:}
    |a|= |(a-b)+b|\leq |a-b|+|b|\\
    \to |a|-|b|\leq |a-b|
    \intertext{Por otro lado:}
    |b|= |(b-a)+a|\leq |b-a|+|a|\\
    \to |b|-|a|\leq |b-a|=|a-b|\\
    \to ||a|-|b||\leq |a-b|
    \end{align}
    \item Si sustituimos $b$ por $-b$ en $(1)$
    \begin{align}
        \to ||a|-|b||\leq |a-(-b)|\\
        \to ||a|-|b||\leq |a+b|
    \end{align}
    \item Si sustituimos en (*) $b$ por $-b$ se tiene: 
    \begin{align}
        |a+(-b)|\leq |a|+|-b|\\
        \to |a-b|\leq |a|+|b|
        \intertext{Conclusión:}
        ||a|-|b||\leq |a\pm b|\leq |a|+|b| \textit{ (Desigualdad triangular)}
    \end{align}
\end{itemize}
\end{remark}

\subsubsection{Propiedad arquímedeada (o de Arquímedes)}
\begin{definition}
Un campo ordenado $\mathbb{F}$ es arquimedeano si $\forall x \in \mathbb{F}\exists n\in \mathbb{Z}^+\ni x<n$.
\end{definition}

\begin{remark}
La clase positiva $P$ de $\mathbb{F}$ es arquimediana si $\forall x\in \mathbb{F}\exists \mathbb{Z}^+\ni n-x\in P.$ 
\end{remark}

\begin{kaobox}[frametitle=Ejercicios]
\begin{enumerate}
    \item Compruebe que el campo $\mathbb{Q}$ es arquimeadeano. 
    \begin{itemize}
        \item ¿Depende este enunciado del orden que se defina en $\mathbb{Q}$?
    \end{itemize}
    \item Presente un ejemplo de un campo ordenado que sea arquimediano. 
\end{enumerate}
\end{kaobox}

\begin{theorem}
Si $\mathbb{F}$ es un campo arquimedeano, entonces: 
\begin{itemize}
    \item Si $y>0$ y $z>0\to\exists n\in \mathbb{Z}^+\ni ny>z$
    \item Si $z>0\to \exists n \in \mathbb{Z}^+\ni 0<\frac{1}{n}<z$
    \item Si $y>0\to \exists n \in \mathbb{Z}^+\ni n-1\leq y< n$
\end{itemize}
\end{theorem}

\begin{proof}
...
\begin{enumerate}
    \item Si $y>0$ y $z>0\ni x:=\frac{z}{y}>0\to \exists n\in \mathbb{Z}^+\ni n>\frac{z}{y}\to ny>z$
    \item Si $z>0\to \frac{1}{z}>0\to \exists n \in \mathbb{Z}^+\ni n>\frac{1}{z}\to \frac{1}{n}<z\to  0<\frac{1}{n}<z$
    \item Sea $y>0\to\exists m\in \mathbb{Z}^+\ni y<m.$
    \begin{center}
        

\tikzset{every picture/.style={line width=0.75pt}} %set default line width to 0.75pt        

\begin{tikzpicture}[x=0.75pt,y=0.75pt,yscale=-1,xscale=1]
%uncomment if require: \path (0,300); %set diagram left start at 0, and has height of 300

%Straight Lines [id:da34781333711164497] 
\draw    (135,152.01) -- (488,153.39) ;
\draw [shift={(490,153.4)}, rotate = 180.22] [color={rgb, 255:red, 0; green, 0; blue, 0 }  ][line width=0.75]    (13.12,-3.95) .. controls (8.34,-1.68) and (3.97,-0.36) .. (0,0) .. controls (3.97,0.36) and (8.34,1.68) .. (13.12,3.95)   ;
\draw [shift={(132,152)}, rotate = 0.22] [fill={rgb, 255:red, 0; green, 0; blue, 0 }  ][line width=0.08]  [draw opacity=0] (10.72,-5.15) -- (0,0) -- (10.72,5.15) -- cycle    ;
%Straight Lines [id:da8197597860941424] 
\draw    (200,141.4) -- (200,152) ;
%Shape: Circle [id:dp09256163312530119] 
\draw  [fill={rgb, 255:red, 0; green, 0; blue, 0 }  ,fill opacity=1 ] (257,152.5) .. controls (257,149.46) and (259.46,147) .. (262.5,147) .. controls (265.54,147) and (268,149.46) .. (268,152.5) .. controls (268,155.54) and (265.54,158) .. (262.5,158) .. controls (259.46,158) and (257,155.54) .. (257,152.5) -- cycle ;
%Shape: Circle [id:dp6277393793069347] 
\draw  [fill={rgb, 255:red, 0; green, 0; blue, 0 }  ,fill opacity=1 ] (323,153.5) .. controls (323,151.57) and (321.43,150) .. (319.5,150) .. controls (317.57,150) and (316,151.57) .. (316,153.5) .. controls (316,155.43) and (317.57,157) .. (319.5,157) .. controls (321.43,157) and (323,155.43) .. (323,153.5) -- cycle ;
%Straight Lines [id:da8933443153155035] 
\draw    (300,141.4) -- (300,152) ;
%Shape: Circle [id:dp3886664444977367] 
\draw  [fill={rgb, 255:red, 0; green, 0; blue, 0 }  ,fill opacity=1 ] (350,153.5) .. controls (350,151.57) and (348.43,150) .. (346.5,150) .. controls (344.57,150) and (343,151.57) .. (343,153.5) .. controls (343,155.43) and (344.57,157) .. (346.5,157) .. controls (348.43,157) and (350,155.43) .. (350,153.5) -- cycle ;
%Shape: Circle [id:dp7285230821464551] 
\draw  [fill={rgb, 255:red, 0; green, 0; blue, 0 }  ,fill opacity=1 ] (376,152.5) .. controls (376,150.57) and (374.43,149) .. (372.5,149) .. controls (370.57,149) and (369,150.57) .. (369,152.5) .. controls (369,154.43) and (370.57,156) .. (372.5,156) .. controls (374.43,156) and (376,154.43) .. (376,152.5) -- cycle ;
%Shape: Circle [id:dp05916074085979284] 
\draw  [fill={rgb, 255:red, 0; green, 0; blue, 0 }  ,fill opacity=1 ] (401,153.5) .. controls (401,151.57) and (399.43,150) .. (397.5,150) .. controls (395.57,150) and (394,151.57) .. (394,153.5) .. controls (394,155.43) and (395.57,157) .. (397.5,157) .. controls (399.43,157) and (401,155.43) .. (401,153.5) -- cycle ;
%Shape: Circle [id:dp7010184215808682] 
\draw  [fill={rgb, 255:red, 0; green, 0; blue, 0 }  ,fill opacity=1 ] (427,151) .. controls (427,148.79) and (425.21,147) .. (423,147) .. controls (420.79,147) and (419,148.79) .. (419,151) .. controls (419,153.21) and (420.79,155) .. (423,155) .. controls (425.21,155) and (427,153.21) .. (427,151) -- cycle ;
%Straight Lines [id:da16881291068353932] 
\draw  [dash pattern={on 0.84pt off 2.51pt}]  (369,83) -- (319.5,150) ;
%Straight Lines [id:da5836547700471342] 
\draw  [dash pattern={on 0.84pt off 2.51pt}]  (374,79) -- (346.5,153.5) ;
%Straight Lines [id:da027764970183191462] 
\draw  [dash pattern={on 0.84pt off 2.51pt}]  (374,79) -- (369,152.5) ;
%Straight Lines [id:da432529211764785] 
\draw  [dash pattern={on 0.84pt off 2.51pt}]  (374,79) -- (397.5,150) ;
%Straight Lines [id:da9746308185178715] 
\draw  [dash pattern={on 0.84pt off 2.51pt}]  (374,79) -- (419,151) ;

% Text Node
\draw (280,113.5) node [anchor=north west][inner sep=0.75pt]    {$n-1$};
% Text Node
\draw (194,156.5) node [anchor=north west][inner sep=0.75pt]    {$0$};
% Text Node
\draw (259,155.6) node [anchor=north west][inner sep=0.75pt]    {$y$};
% Text Node
\draw (313,155.8) node [anchor=north west][inner sep=0.75pt]    {$m$};
% Text Node
\draw (338,155.5) node [anchor=north west][inner sep=0.75pt]    {$n$};
% Text Node
\draw (367,56.5) node [anchor=north west][inner sep=0.75pt]    {$\mathbb{Z}^{+}$};


\end{tikzpicture}
    \end{center}
    \marginnote[-12   pt ]{En matemática, el teorema del buen orden establece que todo conjunto puede ser bien ordenado. Un conjunto X está bien ordenado por un orden estricto si todo subconjunto no vacío de X tiene un elemento mínimo bajo dicho orden. También se conoce como teorema de Zermelo y es equivalente al axioma de elección.}
    Sea $n$ el menor de tales enteros positivos $m$ ($n$ existe por el principio del buen orden). Entonces, $n-1\leq y < n$. 
\end{enumerate} 
\end{proof}

\begin{remark}[Cota superior más pequeña]
Sea $B\subset \mathbb{Q}, B\neq \mathbb{Q}$. Entonces, $B$ es acotado superiormente, si $k\in \mathbb{Q}$. Entonces, $B$ es acotado superiormente si $k\in \mathbb{Q}\ni k\geq b, \forall b\in B$. (En este caso $k$ es cota superior de $B$. 
\end{remark}

\begin{example}
\begin{enumerate}
    \item Sea $\s{a\in \mathbb{Q}\to a<4}$ (este conjunto está acotado superiormente por 4 [pero también 5, 16,... son cotas superiores.]). Por otro lado, $\mathbb{N} \leq \mathbb{Q}$ no es acotado. 
    \item Si $B\subset \mathbb{Q}, B\neq \mathbb{Q}$ y $B$ es acotado superiormente, entonces la cota superior más pequeña de $B$ es un número $k\in \mathbb{Q}\ni$
    \begin{itemize}
        \item $k$ es cota superior.
        \item si $C$ es cota superior de $B$, entonces $C\geq k$. 
    \end{itemize}
    \item Si existe la cota superior más pequeña de $B$, esta es única. \newline 
    Suponga que $k_1$ y $k_2$ son las cotas superiores más pequeñas de $B$. Entonces
    \begin{itemize}
        \item Como $k_1$ es cota superior más pequeña y $k_2$ es cota superior $k_2\geq k_1$. 
        \item Como $k_2$ es cota superior más pequeña y $k_1$ es cota superior $\to k_1\geq k_2\to k_1=k_2$
    \end{itemize}
    \item Considere el conjunto $c=\s{a\in \mathbb{Q}: a\geq 0 \text{ y } a<2}$ Nótese que $C$ está acotado superiormente. En efecto, si $a\in C\to a^2<4\to a<2\to 2$ es cota superior de $C$. 
    \item ¿Es 2 la menor cota superior de $C$? No, considere: $a^2 < -\frac{a}{4}\to a<\frac{3}{2}=1.5$
\end{enumerate}
\end{example}

\begin{kaobox}[frametitle=Para investigar]
\begin{itemize}
    \item Cotas superiores.
    \item Supremo de dos conjuntos = Cota superior. 
\end{itemize}
\end{kaobox}


\section{18 de enero de 2021}

\subsection{Espacios métricos}

\begin{definition}
Espacios Metricos

(1) Sean $x$ un conjunto y $$d: x \times x \rightarrow \mathbb{R}\ni$$
$\forall a, b, c \in x$ satisface:
\begin{enumerate}
\item $(a, b) \geqslant 0 ; d(a, b)=0$ ssi $a=b$.
\item $d(a, b)=d(b, a)$
\item $d(a, b) \leqslant d(a, c)+d(c, d)$, entonces $(x, d)$ es un espacio métrico y $d$ es una métrica sobre $x$ a una
distancia sobre $x$
\end{enumerate}


\end{definition}
\section{21 de enero de 2021}
\begin{example}
$d_2: \mathbb{R}^n\times \mathbb{R}^n \to d_2(x,y)=\sqrt{\suma{i=1}{n}{(x_i-y_i)^"}}$ es métrica. A probar $$d_2(x,y)\leq d_2(x,z)+d_2(z,y)$$
\end{example}

\begin{proof}

\begin{align}
    [d_2(x,z)+d_2(z,y)]^2 = \\
    [\sqrt{\sum(x_i-z_i)^2}
    +
    \sqrt{\sum(z_i-y_i)^2}]\\
    = \sum(x_i-z_i)^2+\sum(z_i-y_i)^2+2[(\sum(x_i-z_i)^2)(\sum (z_i-y_i)^2]^{1/2}\geq\\ \sum(x_i-z_i)^2+\sum(z_i-y_i)^2+2\sum (x_i-z_i)(z_i-y_i)\\
    \marginnote{Cauchy-Schwarz $[(\sum a_ib_i)^2\leq (\sum a_i^2)(\sum b_i^2)]^{1/2}$}
    = \sum [(x_i-z_i)^2+2(x_i-z_i)(z_i-y_i)+(z_i-y_i)^2]\\
    = \sum[(x_i-z_i)+(z_i-y_i)]^2\\
    \to d_2(x,z)+d_2(z,y)\geq d_2(x,y)
\end{align}
\end{proof}

\begin{example}
Considere: $\mathbb{R}^n\times \mathbb{R}^n \to \mathbb{R}\ni$ $$d_\infty= máx\s{|x_i-y_i|: i=1,...,n}$$
$\to d_\infty$ es una métrica de $\mathbb{R}^n$. 
\marginnote{Tenemos $x=(2,3,4)$ y $y=(-1,2,0)$, entonces $d_\infty(x,y)=máx\s{|2-(-1)|,|3-2|,|4-0|}= máx\s{3,1,4}=4$}
\end{example}

\begin{example}
Sea $B([a,b])$ el conjunto de funciones acotadas definidas en $[a,b]$ y de valores reales. También se denota: 
$$l^\infty([a,b])=\s{f:[a,b]\to \mathbb{R}\ni |f(x)|\leq M, M>0}$$

\textbf{Ejemplos}\newline 
Dadas $f,g\in l^{\infty}[a,b]$

\marginnote{$$f(x)=\begin{cases}
1, x\in \mathbb{Q}\\
0, x\in \mathbb{I}
\end{cases}$$
Función de Dirichlet }

\begin{center}
    

\tikzset{every picture/.style={line width=0.75pt}} %set default line width to 0.75pt        

\begin{tikzpicture}[x=0.75pt,y=0.75pt,yscale=-1,xscale=1]
%uncomment if require: \path (0,300); %set diagram left start at 0, and has height of 300

% Plotting does not support converting to Tikz
%Straight Lines [id:da13236771746301135] 
\draw [color={rgb, 255:red, 255; green, 50; blue, 0 }  ,draw opacity=1 ]   (145.8,113) -- (144.8,240) ;
%Straight Lines [id:da30162929751267176] 
\draw [color={rgb, 255:red, 255; green, 50; blue, 0 }  ,draw opacity=1 ]   (518.8,131) -- (516.8,239) ;
%Straight Lines [id:da9538754545706293] 
\draw [color={rgb, 255:red, 255; green, 0; blue, 4 }  ,draw opacity=1 ] [dash pattern={on 4.5pt off 4.5pt}]  (126,106) -- (517.8,106) ;
%Straight Lines [id:da8164736652922968] 
\draw [color={rgb, 255:red, 255; green, 0; blue, 4 }  ,draw opacity=1 ] [dash pattern={on 4.5pt off 4.5pt}]  (124,194) -- (515.8,194) ;
%Straight Lines [id:da18688557822586516] 
\draw [color={rgb, 255:red, 248; green, 231; blue, 28 }  ,draw opacity=1 ]   (210.8,114) -- (209.8,241) ;
%Straight Lines [id:da8090779729750256] 
\draw    (311.96,108) -- (310.84,190) ;
\draw [shift={(310.8,193)}, rotate = 270.78] [fill={rgb, 255:red, 0; green, 0; blue, 0 }  ][line width=0.08]  [draw opacity=0] (8.93,-4.29) -- (0,0) -- (8.93,4.29) -- cycle    ;
\draw [shift={(312,105)}, rotate = 90.78] [fill={rgb, 255:red, 0; green, 0; blue, 0 }  ][line width=0.08]  [draw opacity=0] (8.93,-4.29) -- (0,0) -- (8.93,4.29) -- cycle    ;
%Straight Lines [id:da6781602044348318] 
\draw    (361.8,84) -- (326.34,123.51) ;
\draw [shift={(325,125)}, rotate = 311.90999999999997] [color={rgb, 255:red, 0; green, 0; blue, 0 }  ][line width=0.75]    (10.93,-3.29) .. controls (6.95,-1.4) and (3.31,-0.3) .. (0,0) .. controls (3.31,0.3) and (6.95,1.4) .. (10.93,3.29)   ;

% Text Node
\draw (103,222.1) node [anchor=north west][inner sep=0.75pt]    {$0$};
% Text Node
\draw (288,224.1) node [anchor=north west][inner sep=0.75pt]    {$x$};
% Text Node
\draw (139,242.1) node [anchor=north west][inner sep=0.75pt]    {$a$};
% Text Node
\draw (510,254.1) node [anchor=north west][inner sep=0.75pt]    {$b$};
% Text Node
\draw (211,255.1) node [anchor=north west][inner sep=0.75pt]    {$x$};
% Text Node
\draw (353,57.1) node [anchor=north west][inner sep=0.75pt]    {$d_{_{\infty }}( f,g)$};


\end{tikzpicture}
\end{center}

$$\to d_\infty(f,g)=sup_{x\in[a,b]}\s{|f(x)-g(x)|}$$, la cual es una métrica en $l^{\infty}[a,b]$ y se llama métrica o distancia del supremo.
\end{example}

\begin{example}
Sea $C[a,b]$ el conjunto de funciones continuas sobre el $[a,b]$ con valores reales. Entonces, si $f,g\in C[a,b]$, se tiene: la métrica $d(f,g)=\int_a^b |f(x)-g(x)|dx$ sobre $C[a,b]$
\end{example}

\begin{definition}[Norma]
Suponga que $V$ es un espacio vectorial sobre el campo $\mathbb{F}(\mathbb{R} o \mathbb{C})$ y que $$\norm{\cdot}: V\to \mathbb{R}\ni$$
$\forall x,y\in V$ y $\alpha\in \mathbb{F}$, se cumplen:
\begin{itemize}
    \item $\norm{x}\geq 0, \norm{x}=0$ ssi $x=0$
    \item $\norm{\alpha x}=|\alpha|\cdot \norm{x}$
    \item $\norm{x+y}\leq \norm{x}+\norm{y}$
    
    Entonces, $\norm{\cdot}$ es una norma sobre $V$, y decimos que $(V,\norm{\cdot})$ es un espacio normado.
\end{itemize}
\end{definition}

\begin{remark}
Sea $V$ un espacio vectorial normado. Entonces, considere: 
$$d:V\times V\to \mathbb{R}\ni$$
$$d(x,y)=\norm{x-y}$$
Nótese que: \begin{itemize}
    \item $d(x,y)=\norm{x.y}\geq 0;$\newline 
    Si $x=y\to d(x,y)=\norm{x-y}=0$
    \newline 
    Si $d(x,y)=\norm{x-y}=0\to x-y=0\to x=y$
    \item $d(x,y)=\norm{x-y}=\norm{-(y-x)}=|-1|\norm{y-x}= \norm{y-x}=d(y,x)$
    \item $d(x,y)=\norm{x-y}=\norm{(x-z)+(z-y)}\leq \norm{x-z}+\norm{a}$
    \newline $d(x,y)=\norm{x-y}$ es una métrica sobre $V$. Esta es la métrica inducida por la norma. 
\end{itemize}
\end{remark}


\subsection{Topología de $\mathbb{R}$ ($\mathbb{R}^n$)}
\begin{definition}
Sea $x$ un conjunto no vacío. Una familia de subconjunto $\tau$ de $x$ es una topología sobre $x$ si: 
\begin{itemize}
    \item $\emptyset, x\in \tau$
    \item Cualquier familia $\s{A_i}$ de elementos de $\tau$ es t.q. $\bigcup_i A_i\in \tau$
    \item Si $A_1,A_2\in \tau\to A_1\cap A_2\in \tau$. A los elementos de $\tau$ se les llama abiertos de $x$. 
\end{itemize}
\end{definition}
\section{25 de enero de 2021}

\subsection{Topología de Espacios Métricos}
\begin{definition}
Sea $(M,d)$ un espacio métrico. 
\begin{enumerate}
    \item La \textbf{Bola abierta} de centro en $a$ y radio $r$ es: $B_r(a)\{x\in M\ni d(x,a)<r\}$
    \item La \textbf{Bola cerrada} de centro en $a$ y radio $r$ es $\Bar{B}_r(a)=B_r[a]=\{x\in M\ni d(x,a)\leq r\}$
\end{enumerate}
\end{definition}

\begin{remark}
\begin{enumerate}
    \item En el caso que $M=\mathbb{R}$ y $d(x,y)=|x-y|,\forall x,y\in\mathbb{R},\implies B_r(a)=\{x\in\mathbb{R}:|x-a|<r\} =\{x\in\mathbb{R}: a-r<x<a+r\}=(a-r,a+r)$. Que es análogo para $[a-r,a+r]$
    \item En el caso de $\mathbb{R}^2$ o $\mathbb{C}$, las bolas se llaman discos. 
    \item Un subconjunto de un espacio métrico es acotado si está contenido en una bola; $A\subset M$ es acotado, si $\exists r>0,a\in M\ni A\subset B_r (a)$. (i.e. $d(x,a)<r,\forall x\in A)$.
    \item Las bolas abiertas y cerradas son acotadas. 
\end{enumerate}
\end{remark}
\marginnote{Los elementos de una topología se le llama abiertos.}
\begin{definition}
\begin{enumerate}
    \item  Un subconjunto $u$ del espacio métrico $M$ es un abierto, si $\forall a\in u\exists r>0 \ni B_r(a)\subset u$
    \begin{center}
        

\tikzset{every picture/.style={line width=0.75pt}} %set default line width to 0.75pt        

\begin{tikzpicture}[x=0.75pt,y=0.75pt,yscale=-1,xscale=1]
%uncomment if require: \path (0,300); %set diagram left start at 0, and has height of 300

%Shape: Circle [id:dp5694297831696636] 
\draw  [dash pattern={on 4.5pt off 4.5pt}] (162,137.9) .. controls (162,84.38) and (205.38,41) .. (258.9,41) .. controls (312.42,41) and (355.8,84.38) .. (355.8,137.9) .. controls (355.8,191.42) and (312.42,234.8) .. (258.9,234.8) .. controls (205.38,234.8) and (162,191.42) .. (162,137.9) -- cycle ;
%Shape: Circle [id:dp0054591279969846696] 
\draw  [color={rgb, 255:red, 255; green, 0; blue, 0 }  ,draw opacity=1 ][dash pattern={on 4.5pt off 4.5pt}] (262,92.9) .. controls (262,76.94) and (274.94,64) .. (290.9,64) .. controls (306.86,64) and (319.8,76.94) .. (319.8,92.9) .. controls (319.8,108.86) and (306.86,121.8) .. (290.9,121.8) .. controls (274.94,121.8) and (262,108.86) .. (262,92.9) -- cycle ;
%Shape: Circle [id:dp6755751520546592] 
\draw  [fill={rgb, 255:red, 65; green, 117; blue, 5 }  ,fill opacity=1 ] (280.8,92.9) .. controls (280.8,87.32) and (285.32,82.8) .. (290.9,82.8) .. controls (296.48,82.8) and (301,87.32) .. (301,92.9) .. controls (301,98.48) and (296.48,103) .. (290.9,103) .. controls (285.32,103) and (280.8,98.48) .. (280.8,92.9) -- cycle ;
%Straight Lines [id:da8296107228701722] 
\draw    (301,92.9) -- (319.8,92.9) ;

% Text Node
\draw (335,50.1) node [anchor=north west][inner sep=0.75pt]  [font=\Large]  {$u$};
% Text Node
\draw (289,102.1) node [anchor=north west][inner sep=0.75pt]    {$a$};


\end{tikzpicture}
    \end{center}
    \item La familia $\theta$ de todos los subconjuntos abiertos de $M$ es la \textbf{topología de M}, y el par $(M,\theta)$ es espacio topológico asociado al métrico $M$. 
\end{enumerate}
\end{definition}

\begin{remark}
En el caso de $\mathbb{R}^n$ se dice que se tiene el espacio topológico Euclidiano $\mathbb{R}^n$. 
\end{remark}

\begin{example}
Algunos ejemplos...
\begin{enumerate}
    \item $\mathbb{R}^n$ es abierto.  En efecto, $B_1(x)\subset \mathbb{R}^n,\forall x\in\mathbb{R}^n$. 
    \item $G=\{x\in\mathbb{R}: 0<x<1\}$ es abierto, pero $F=\{x\in\mathbb{R}:0\leq x<1\}$ no lo es. 
    \item $G=\{(x,y)\in\mathbb{R}^2\ni x^2+y^2<1\}$ es abierto y $F=\{(x,y)\in\mathbb{R}\ni x^2+y^2\leq 1\}$ no es abierto. 
    \item $G=\{(x,y)\in\mathbb{R}^2\ni 0<x<1,y=0\}$ no es abierto de $\mathbb{R}^2$. $F=\{(x,y)\in\mathbb{R}^2\ni 0<x<1\}$ es abierto de $\mathbb{R}^2$
    \item $\emptyset$ es abierto. 
\end{enumerate}
\end{example}

\begin{proposition}
Una bola abierta es abierto. 
\end{proposition}

\begin{proof}
\begin{center}


\tikzset{every picture/.style={line width=0.75pt}} %set default line width to 0.75pt        

\begin{tikzpicture}[x=0.75pt,y=0.75pt,yscale=-1,xscale=1]
%uncomment if require: \path (0,300); %set diagram left start at 0, and has height of 300

%Shape: Ellipse [id:dp6476242320129313] 
\draw  [dash pattern={on 4.5pt off 4.5pt}] (211,102.9) .. controls (211,49.6) and (256.12,6.4) .. (311.79,6.4) .. controls (367.45,6.4) and (412.58,49.6) .. (412.58,102.9) .. controls (412.58,156.2) and (367.45,199.4) .. (311.79,199.4) .. controls (256.12,199.4) and (211,156.2) .. (211,102.9) -- cycle ;
%Straight Lines [id:da2972831506157503] 
\draw    (309.81,104.16) -- (412.58,102.9) ;
%Shape: Ellipse [id:dp05468073252937922] 
\draw  [fill={rgb, 255:red, 255; green, 0; blue, 0 }  ,fill opacity=1 ] (299.44,104.16) .. controls (299.44,101.42) and (301.76,99.19) .. (304.62,99.19) .. controls (307.49,99.19) and (309.81,101.42) .. (309.81,104.16) .. controls (309.81,106.9) and (307.49,109.13) .. (304.62,109.13) .. controls (301.76,109.13) and (299.44,106.9) .. (299.44,104.16) -- cycle ;
%Shape: Ellipse [id:dp2946667421288577] 
\draw  [color={rgb, 255:red, 255; green, 0; blue, 0 }  ,draw opacity=1 ][dash pattern={on 4.5pt off 4.5pt}] (255.47,51.18) .. controls (255.47,29.5) and (273.83,11.92) .. (296.47,11.92) .. controls (319.12,11.92) and (337.48,29.5) .. (337.48,51.18) .. controls (337.48,72.86) and (319.12,90.44) .. (296.47,90.44) .. controls (273.83,90.44) and (255.47,72.86) .. (255.47,51.18) -- cycle ;
%Straight Lines [id:da862904540649499] 
\draw [color={rgb, 255:red, 255; green, 0; blue, 0 }  ,draw opacity=1 ]   (296.47,51.18) -- (281.65,16.65) ;

% Text Node
\draw (387.24,9.8) node [anchor=north west][inner sep=0.75pt]    {$B_{r}( a)$};
% Text Node
\draw (305.46,105.41) node [anchor=north west][inner sep=0.75pt]    {$a$};
% Text Node
\draw (366.39,106.2) node [anchor=north west][inner sep=0.75pt]    {$r$};
% Text Node
\draw (298.87,38.45) node [anchor=north west][inner sep=0.75pt]    {$x$};
% Text Node
\draw (270,217.1) node [anchor=north west][inner sep=0.75pt]    {$p=r-d( a,x)$};
% Text Node
\draw (274,43.1) node [anchor=north west][inner sep=0.75pt]  [color={rgb, 255:red, 74; green, 144; blue, 226 }  ,opacity=1 ]  {$y$};


\end{tikzpicture}
\end{center}
Sea $x\in B_r(a)$ y considere la bola centrada en $x$ y de radio $r-d(a,x)$. A probar: $B_{r-d(a,x)}(x)\subset B_r(a)$.\newline\newline 
Sea $y\in B_{r-d(a,x)}(x)$. Entonces,
\begin{align}
    d(a,y)&\leq d(a,x)+d(x,y)
    &< d(a,x)+[r-d(a,x)]
    &r
\end{align}
$\implies y\in B_r(a)$
\end{proof}

\begin{theorem}
\begin{enumerate}
    \item $\emptyset$ y $\mathbb{R}^n$ son abiertos. 
    \item La intersección de dos abiertos de $\mathbb{R}^n$\footnote{Por inducción se deduce que la intersección de finita de abiertos es abierto.} es abierto de $\mathbb{R}^n$
    \item La unión de cualquier colección de abiertos es un abierto de $\mathbb{R}^n$
\end{enumerate}
\end{theorem}

\begin{proof}
\begin{enumerate}
    \item OK. 
    \item Sea $A$ y $B$ abiertos de $\mathbb{R}^n$. A probar: $A\cap B$ es abierto. Sea $x\in A\cap B$, entonces: \newline 
    $\implies x\in A$, abierto, $\implies \exists r>0\ni d(x,z)<r$, para $z\in A$. \newline 
    $x\in B$, abierto, $\implies \exists r'>0\ni d(x,w)<r$, para $w\in B$. 
    
    \begin{center}
        

\tikzset{every picture/.style={line width=0.75pt}} %set default line width to 0.75pt        

\begin{tikzpicture}[x=0.75pt,y=0.75pt,yscale=-1,xscale=1]
%uncomment if require: \path (0,300); %set diagram left start at 0, and has height of 300

%Shape: Circle [id:dp6379702365937482] 
\draw   (46,121.9) .. controls (46,76.67) and (82.67,40) .. (127.9,40) .. controls (173.13,40) and (209.8,76.67) .. (209.8,121.9) .. controls (209.8,167.13) and (173.13,203.8) .. (127.9,203.8) .. controls (82.67,203.8) and (46,167.13) .. (46,121.9) -- cycle ;
%Shape: Circle [id:dp3871343130648992] 
\draw   (138,123.9) .. controls (138,78.67) and (174.67,42) .. (219.9,42) .. controls (265.13,42) and (301.8,78.67) .. (301.8,123.9) .. controls (301.8,169.13) and (265.13,205.8) .. (219.9,205.8) .. controls (174.67,205.8) and (138,169.13) .. (138,123.9) -- cycle ;
%Shape: Circle [id:dp6625358626104291] 
\draw  [color={rgb, 255:red, 245; green, 12; blue, 12 }  ,draw opacity=1 ][dash pattern={on 4.5pt off 4.5pt}] (107,124.9) .. controls (107,105.07) and (123.07,89) .. (142.9,89) .. controls (162.73,89) and (178.8,105.07) .. (178.8,124.9) .. controls (178.8,144.73) and (162.73,160.8) .. (142.9,160.8) .. controls (123.07,160.8) and (107,144.73) .. (107,124.9) -- cycle ;
%Shape: Circle [id:dp6083087375414079] 
\draw  [color={rgb, 255:red, 74; green, 144; blue, 226 }  ,draw opacity=1 ][dash pattern={on 4.5pt off 4.5pt}] (148.1,123.9) .. controls (148.1,104.07) and (164.17,88) .. (184,88) .. controls (203.83,88) and (219.9,104.07) .. (219.9,123.9) .. controls (219.9,143.73) and (203.83,159.8) .. (184,159.8) .. controls (164.17,159.8) and (148.1,143.73) .. (148.1,123.9) -- cycle ;
%Straight Lines [id:da8851356973361522] 
\draw [color={rgb, 255:red, 208; green, 2; blue, 27 }  ,draw opacity=1 ]   (138,123.9) -- (114.8,145.4) ;
%Straight Lines [id:da800570752888615] 
\draw [color={rgb, 255:red, 74; green, 144; blue, 226 }  ,draw opacity=1 ]   (215,112.9) -- (193.8,126.4) ;

% Text Node
\draw (57,36.1) node [anchor=north west][inner sep=0.75pt]    {$A$};
% Text Node
\draw (278,34.1) node [anchor=north west][inner sep=0.75pt]    {$B$};
% Text Node
\draw (118,113.1) node [anchor=north west][inner sep=0.75pt]    {$r$};
% Text Node
\draw (184,107.1) node [anchor=north west][inner sep=0.75pt]    {$r'$};
% Text Node
\draw (157,113.1) node [anchor=north west][inner sep=0.75pt]    {$x$};


\end{tikzpicture}
    \end{center}
    
    $\implies$ Hagamos $r=min\{r,r'\}\implies$ si $y\in\mathbb{R}\ni d(x,y)<r\implies y\in A$ y $y\in B\implies y\in A\cap B\implies A\cap B$ es abierto en $\mathbb{R}^n$
    
\item \marginnote{$\bigcup_{i=1}^n A_i$.} Sea $\{G_\alpha\}$ una colección cualquiera de abiertos de $\mathbb{R}^n$ \marginnote{$\{G_\alpha:\alpha\in I\}$}, y sea $G=\bigcup_\alpha G_{\alpha} $. Si $x\in G\implies x\in G_\lambda$, para algún $\lambda$. Como $G_{\lambda}$ es abierto $\implies \exists r>0\ni B_r(x)\subset G_\lambda \subset $ 
    
\end{enumerate}
\end{proof}

\begin{remark}
La intersección de una colección infinita de abiertos, no necesariamente es abierta. En efecto, considere: $$A_n=\{x\in\mathbb{R}\ni -\frac{1}{n}<x<1+\frac{1}{n}\}, n\in\mathbb{Z}^+$$
$$A_1=(-1,2)$$
$$A_2=(-\frac{1}{2},\frac{3}{2})$$
\marginnote{Los $A_n$ son abiertos (por ser bolas abiertas en $\mathbb{R}$)}
$\implies A =\bigcap_{n=1}^\infty A_n=[0,1]$\footnote{¿Por qué cerrado?}
\end{remark}

\begin{definition}
Un subconjunto $\mathbb{F}$ en el métrico $(M,d)$ es cerrado si $\mathbb{F}^c$ es abierto. 
$[0,1]=(-\infty,0)\cup (1,\infty)\implies [0,1]$ es cerrado. 
\end{definition}

\begin{example}
\begin{enumerate}
    \item $\emptyset$ es abierto $\implies \emptyset^c=\mathbb{R}^n$ es cerrado. 
    \item $\mathbb{R}^n$ es abierto $\implies (\mathbb{R}^n)^c=\emptyset$ es cerrado. $\implies\emptyset$ y $\mathbb{R}^n$ son abiertos y cerrados. 
    \item $[0,1)$ no es abierto ni cerrado. 
    
\end{enumerate}
\end{example}
\section{28 de enero de 2021}

\begin{definition}
Sea $x\in M$ (espacio métrico), entonces cualquier conjunto que contiene un abierto $A\ni x\in A$ es una vecindad de $x$.

\begin{center}
    

\tikzset{every picture/.style={line width=0.75pt}} %set default line width to 0.75pt        

\begin{tikzpicture}[x=0.75pt,y=0.75pt,yscale=-1,xscale=1]
%uncomment if require: \path (0,300); %set diagram left start at 0, and has height of 300

%Shape: Rectangle [id:dp5325951477231496] 
\draw   (31,43.81) -- (201.19,43.81) -- (201.19,180.4) -- (31,180.4) -- cycle ;
%Shape: Rectangle [id:dp6308620272255319] 
\draw  [color={rgb, 255:red, 74; green, 144; blue, 226 }  ,draw opacity=1 ] (47.3,59.84) -- (155.54,59.84) -- (155.54,149.14) -- (47.3,149.14) -- cycle ;
%Shape: Ellipse [id:dp09032950096806747] 
\draw  [dash pattern={on 4.5pt off 4.5pt}] (58.71,88.14) .. controls (58.71,74.9) and (69.62,64.17) .. (83.08,64.17) .. controls (96.54,64.17) and (107.45,74.9) .. (107.45,88.14) .. controls (107.45,101.37) and (96.54,112.1) .. (83.08,112.1) .. controls (69.62,112.1) and (58.71,101.37) .. (58.71,88.14) -- cycle ;
%Shape: Ellipse [id:dp677270653778341] 
\draw  [fill={rgb, 255:red, 255; green, 0; blue, 0 }  ,fill opacity=1 ] (68.57,88.14) .. controls (68.57,84.2) and (71.82,81) .. (75.83,81) .. controls (79.83,81) and (83.08,84.2) .. (83.08,88.14) .. controls (83.08,92.08) and (79.83,95.27) .. (75.83,95.27) .. controls (71.82,95.27) and (68.57,92.08) .. (68.57,88.14) -- cycle ;
%Straight Lines [id:da8462026494214948] 
\draw    (42,231) -- (197.8,229.4) ;
%Straight Lines [id:da36445047109408835] 
\draw    (70.8,216.4) -- (70.8,244.4) ;
%Straight Lines [id:da354987556770879] 
\draw    (89.8,215.4) -- (89.8,243.4) ;
%Straight Lines [id:da777575302939591] 
\draw    (111.8,216.4) -- (111.8,244.4) ;
%Straight Lines [id:da7237425422785222] 
\draw    (129.8,216.4) -- (129.8,244.4) ;
%Straight Lines [id:da6947724696435326] 
\draw    (149.8,215.4) -- (149.8,243.4) ;

% Text Node
\draw (208.65,39.12) node [anchor=north west][inner sep=0.75pt]    {$M$};
% Text Node
\draw (83.68,75.19) node [anchor=north west][inner sep=0.75pt]    {$x$};
% Text Node
\draw (101.76,105.41) node [anchor=north west][inner sep=0.75pt]    {$A$};
% Text Node
\draw (156.04,55.15) node [anchor=north west][inner sep=0.75pt]    {$V$};
% Text Node
\draw (66,248.1) node [anchor=north west][inner sep=0.75pt]    {$0$};
% Text Node
\draw (85,247.1) node [anchor=north west][inner sep=0.75pt]    {$1$};
% Text Node
\draw (106,248.1) node [anchor=north west][inner sep=0.75pt]    {$2$};
% Text Node
\draw (126,247.1) node [anchor=north west][inner sep=0.75pt]    {$3$};
% Text Node
\draw (146.8,247.5) node [anchor=north west][inner sep=0.75pt]    {$4$};
% Text Node
\draw (58,210.9) node [anchor=north west][inner sep=0.75pt]  [font=\large,color={rgb, 255:red, 255; green, 0; blue, 0 }  ,opacity=1 ] [align=left] {[ \ \ \ \ \ \ \ \ \ \ \ \ \ \ \ \ \ \ )};
% Text Node
\draw (89,213.9) node [anchor=north west][inner sep=0.75pt]  [color={rgb, 255:red, 126; green, 211; blue, 33 }  ,opacity=1 ] [align=left] {( \ \ \ \ \ )};


\end{tikzpicture}
\end{center}
\end{definition}

\begin{definition}
Un punto $x\in M$ es un punto interior de un conjunto $A\subseteq M$, si $A$ es una vecindad de $x$. 
\end{definition}

\begin{example}
\begin{itemize}
    \item $[0,1], x=0$ y $x=1$ no son puntos interiores. El resto de puntos de $[0,1)$ son puntos interiores de $[0,1]$
    \item $I=(0,1)$, todos son puntos interiores. 
    \item $\mathbb{R}\cap \mathbb{Z}\subseteq\mathbb{R}$
    %grafica
    $$\implies \mathbb{R}\cap \mathbb{Z}$$
    no tiene puntos interiores. 
    \marginnote{A cualquier conjunto no vacío se le puede dotar de una métrica. 
    $A,d\implies d(x,y)\begin{cases}1,x\neq 1\\ 0,x=0\end{cases}$}
\end{itemize}
\end{example}

\marginnote{Punto interior, por lo menos una bola abierta contenida en $A$}
\begin{definition}
Un punto $x\in M$ es un punto de acumulación (cluster) o punto límite de un conjunto $A\subseteq M$, si cada vecindad de $x$ contiene al menos un punto de $A$ diferente de $x$. Es decir, si 
$$B_r(x)-\{x\})\cap A\neq \emptyset,\forall r>0$$
\begin{example}
$A=\{1,1/2,1/3,...,1/n,...\}\subseteq \mathbb{R}\implies x=0$ es un punto de acumulación de $A$. 
%Grafico 
\end{example}
\end{definition}

\begin{definition}
\begin{enumerate}
    \item El conjunto de todos los puntos interiores de $A$ se llama interior de $A$. (Notación: $A^{0}$ o $int(A)$. %Grafico 
    Es decir, $$int(A)=\bigcup_{U\subset A, \text{u es abierto}}\mathcal{U}$$
    i.e. $int(A)$ es el abierto más grande contenido. \begin{example}
\begin{enumerate}
    \item $int[0,1]=(0,1)$
    \item $int \mathbb{R}\cap \mathbb{Z}=\emptyset$
    \item $int\mathbb{R}^n=\mathbb{R}^n$
    \item $A$ es un conjunto abierto ssi $A=int(A)$
\end{enumerate}
\end{example}
\item La cerradura de $A$ es el conjunto: $$\Bar{A}:=\bigcap_{A\subset F, \text{F cerrado}}$$\begin{remark}
\begin{enumerate}
    \item $\Bar{A}$ es cerrado
    \item $\Bar{A}$ es el cerrado más pequeño que contiene a $A$.
    \item $A$ es cerrado ssi $A=\Bar{A}$
    \item Si $F$ es un cerrado que contiene a $A\implies A\subset \Bar{A}\subset F$
\end{enumerate}
\end{remark}
\item La frontera de $A$ (denotada $bd(A)$ o $\partial A$) Se define: 
$$\partial A:= \Bar{A}-int(A)$$
\begin{example}
Sea$I=[0,1]\implies \Bar{I}=[0,1]\implies (0,1)\implies \partial A=\Bar{I}-int(A)=\{0,1\}$
\end{example}
\item El conjunto de todos los puntos de acumulación de un conjunto $A$ se llama conjunto derivado de $A$. Notación: $A'$
\end{enumerate}
\end{definition}


\begin{proposition}
\begin{enumerate}
    \item Si $A\subset B\implies A'\subset B'$
    \begin{proof}
    Sea $x\in A'$ (i.e. $x$ es un punto de acumulación de $A$). $\implies \forall $ abierto $G\ni x\in G$, se tiene que $$(G-\{x\})\cap A\neq \emptyset $$
    \marginnote{$$A\subset B$$ 
    $$C\subset D$$
    $$\implies A\cap C\subset B\cap D$$}
    Como $A\subset B\implies (G-\{x\}\cap A\subset (G-\{x\})\cap B$
    
    $$\implies \emptyset\neq (G-\{x\})\cap A\subset (G-\{x\})\cap B$$
    $$(G-\{x\})\cap B\neq \empty,\forall G\ni x\in G\implies x\in B'$$
    \end{proof}
    
    \item $(A\cup B)'=A'\cup B'$
    \begin{proof}
    \begin{itemize}
        \item (De ida) Sabemos que $A\subset A\cup B$ y $B\subset A\cup B$ $\implies A'\subset (A\cup B)'$ y $B'\subset (A\cup B)'$
        $$\implies A'\cup B'\subset (A\cup B)'$$
        \item (De regreso) $(A\cup B)'\subset A'\cup B'$\newline 
        $\longleftrightarrow$ Si $x\in (A\cup B)'\implies x\in A'\cup B'$\newline 
        $\Longleftrightarrow$ Si $x\not\in A'\cup B'\implies x\not\in (A\cup B)'$\newline 
        $\implies$ Como $x\not\in A'\implies\exists G$, abierto, tal que $G\cap A\subset \{x\}$\footnote{$G\cap A=\emptyset$ o $G\cap A=\{x\}$ (no)}\newline 
        Como $x\not\in B'\implies \exists H$, abierto, tal que $H\cap A\subset \{x\}$. Nótese que $G\cap H$ es abierto. Entonces, $x\in G\cap H$, y $(G\cap H)\cap (A\cup B)=(G\cap H\cap A)\cup (G\cap H\cap B)\subset \{x\}\cup \{x\}=\{x\}$\newline 
        $\implies x\not\in (A\cup B)'\implies$ si $x\not\in A'\cup B'\implies x\in (A\cup B)'\implies (A\cup B)\subset A'\cup B'\implies (A\cup B)'=A'\cup B'$
        
        
    \end{itemize}
    \end{proof}
\end{enumerate}
\end{proposition}



\section{1 de febrero de 2021}

\begin{definition}
\begin{itemize}
    \item Bola abierta $Br(a)=\{x\in M\ni d(x,a)<r\},\quad r>0$
    \item Abierto: $Br()$ o $\bigcup,\bigcap$
    \item Vecindad $x$ abierta: $A\ni A$
\end{itemize}
\end{definition}

\begin{proposition}
$A$ es cerrado ssi $A'\subset A$\marginnote{Un conjunto es cerrado ssi contiene a sus puntos de acumulación.}
\end{proposition}
\begin{proof}
\begin{itemize}
    \item ($\to$)\marginnote{A probar: $A'\subset A\Longleftrightarrow$ si $x\in A'\implies x\in A\Longleftrightarrow $ si $x\not\in A\implies x\not\in A'$} Sea $A$ cerrado y sea $p\not\in A\implies p\in A^c$, pero $A^c$ es un abierto $\ni p\in A^c$ y $A\cup A^c=\emptyset\implies p\in\not A'\implies A\implies A'\subset A$
    \item ($\gets$) A probar: $A'\subset A\implies A$ es cerrado ($\Longleftrightarrow A^c$ es abierto). Suponga que $A'\subset A$ 
    y sea $p\in A^c\implies p\not\in A'\implies \exists G$, abierto, tal que: $p\in G$ y $$(G-\{p\})\cap A=\emptyset$$
    Como $p\not\in A\implies G\cap A=\emptyset\implies G\subset A^c$. Entonces si $p\in A^c\exists $ abierto $G\ni p\in G\subset A^c\implies A^c$ es abierto. 
    \begin{center}
        

\tikzset{every picture/.style={line width=0.75pt}} %set default line width to 0.75pt        

\begin{tikzpicture}[x=0.75pt,y=0.75pt,yscale=-1,xscale=1]
%uncomment if require: \path (0,300); %set diagram left start at 0, and has height of 300

%Shape: Ellipse [id:dp37154978089883683] 
\draw   (100,111.6) .. controls (100,75.37) and (152.2,46) .. (216.6,46) .. controls (281,46) and (333.2,75.37) .. (333.2,111.6) .. controls (333.2,147.83) and (281,177.2) .. (216.6,177.2) .. controls (152.2,177.2) and (100,147.83) .. (100,111.6) -- cycle ;
%Shape: Ellipse [id:dp6206196527695439] 
\draw  [dash pattern={on 4.5pt off 4.5pt}] (149,98.1) .. controls (149,84.79) and (174.34,74) .. (205.6,74) .. controls (236.86,74) and (262.2,84.79) .. (262.2,98.1) .. controls (262.2,111.41) and (236.86,122.2) .. (205.6,122.2) .. controls (174.34,122.2) and (149,111.41) .. (149,98.1) -- cycle ;
%Curve Lines [id:da7080138380886729] 
\draw    (284.2,114.2) .. controls (323.8,84.5) and (181.1,131.25) .. (217.06,103.08) ;
\draw [shift={(218.2,102.2)}, rotate = 503.13] [color={rgb, 255:red, 0; green, 0; blue, 0 }  ][line width=0.75]    (10.93,-3.29) .. controls (6.95,-1.4) and (3.31,-0.3) .. (0,0) .. controls (3.31,0.3) and (6.95,1.4) .. (10.93,3.29)   ;
%Curve Lines [id:da88098585793335] 
\draw    (136.71,132.07) .. controls (121.26,82.89) and (176.93,160.3) .. (162.9,113.65) ;
\draw [shift={(162.45,112.2)}, rotate = 432.56] [color={rgb, 255:red, 0; green, 0; blue, 0 }  ][line width=0.75]    (10.93,-3.29) .. controls (6.95,-1.4) and (3.31,-0.3) .. (0,0) .. controls (3.31,0.3) and (6.95,1.4) .. (10.93,3.29)   ;

% Text Node
\draw (301,44.1) node [anchor=north west][inner sep=0.75pt]    {$A^{c}$};
% Text Node
\draw (197,84.1) node [anchor=north west][inner sep=0.75pt]    {$p$};
% Text Node
\draw (291,115.1) node [anchor=north west][inner sep=0.75pt]    {$G$};
% Text Node
\draw (124.32,138.74) node [anchor=north east][inner sep=0.75pt]  [xscale=-1]  {$Br( p) =G$};


\end{tikzpicture}
    \end{center}
\end{itemize}
\end{proof}

\begin{proposition}
Si $F$ es un superconjunto cerrado de cualquier conjunto $A$, entonces $A'\subset F$ 
\end{proposition}
\begin{proof}
Sabemos que $F$ es cerrado y $A\subset F$. Como $A\subset F\implies A'\subset F'$. Como $F$ es cerrado, entonces $F'\subset F\implies A'\subset F$
\end{proof}

\begin{proposition}
$A\cup A'$ es cerrado. 
\end{proposition}
\begin{proof}

Se asume que $A^{\prime}$ que significa el conjunto de elementos limitantes con $A$. Para un conjunto $B$, Se denota cl $(B)$ como cerradura.
Desde que $\operatorname{cl}\left(A \cup A^{\prime}\right) \supseteq \operatorname{cl}(A) \supseteq A^{\prime},$ miramos
$$
A \cup A^{\prime} \subseteq \operatorname{cl}\left(A \cup A^{\prime}\right)
$$
entonces, se necesita probar que $A \cup A^{\prime}$ es cerrado.
Suponemos $x$ es un punto límite de $A \cup A^{\prime} .$ Si $x \in A,$ entonces no hay nada que probar, entonces suponemos $x \notin A$. Decimos que $U$ es un vecindario de $x$; entonces existe $y \in A \cup A^{\prime}, y \neq x,$ con $y \in A \cup A^{\prime} .$ Si $y \in A,$ entonces está comprobado. De otra forma $y \in A^{\prime} \backslash A$; si tomamos un vecindario $V$ de $y, V \subseteq U$; entonces hay $z \in A \cap V$, $z \neq y ;$ por lo que $z \in A \cap U, z \neq x$
\end{proof}

\begin{proposition}
$\Bar{A}=A\cup A'$
\end{proposition}
\begin{proof}
\begin{itemize}
    \item ($\supseteq$) Sabemos que $A\subset \Bar{A}$.\marginnote{$A\subset B\implies A'\subset B'$} Por otra parte, $\implies  A'\subset(\Bar{A})'\implies \Bar{A}\implies A'\subset \Bar{A}\implies A\cup A'\subset \Bar{A}$.
    \item ($\subseteq$) A probar: $\Bar{A}\subset A\cup A'$. Entonces, $A\subset A\cup A'\implies A\subset \Bar{A}\subset A\cup A'$
\end{itemize}
\end{proof}

\begin{proposition}
Si $A\subset B\implies \Bar{A}\subset \Bar{B}$
\end{proposition}

\begin{proof}
Si $A\subset B\implies A'\subset B'\implies A\cup B'\subset B\cup B'\implies \Bar{A}\subset \Bar{B}$
\end{proof}

\begin{proposition}
$\overline{A\cup B}=\Bar{A}\cup \Bar{B}$
\end{proposition}

\begin{proof}
\begin{itemize}
    \item $(\subseteq)$ A probar $\overline{A\cup B}\subset \Bar{A}\cup \Bar{B}$. Sabemos que $A\subset \Bar{A}$ y $B\subset \Bar{B}\implies A\cup B\subset \Bar{A}\cup \Bar{B}\implies A\cup B\subset \overline{A\cup B}\subset \Bar{A}\cup \Bar{B}$
    \item $A\subset A\cup B \implies \Bar{A}\subset \overline{A\cup B}$\\
          $B\subset A\cup B \implies \Bar{B}\subset \overline{A\cup B}$\\
          $\implies \Bar{A}\cup\Bar{B}\subset \overline{A\cup B}\cup \overline{A\cup B}\implies \Bar{A}\cup \Bar{B}\subset \overline{A\cup B}$
\end{itemize}
\end{proof}

\begin{remark}[Axiomas de Kuratowski]
\marginnote{Propone: Construir una topología de cerrados a partir de $k_1-k_4$}
\begin{itemize}
    \item $K_1$ : $\Bar{\emptyset}=\emptyset$
    \item $K_2$ : $A\subset\Bar{A}$
    \item $K_3$ : $\Bar{\Bar{A}}=\Bar{A}$
    \item $K_4$ : $\overline{A\cup B}=\Bar{A}\cup \Bar{B}$
    \end{itemize}
\end{remark}

\begin{definition}
$$\frac{f(A)}{f(C)}$$
\begin{center}
    

\tikzset{every picture/.style={line width=0.75pt}} %set default line width to 0.75pt        

\begin{tikzpicture}[x=0.75pt,y=0.75pt,yscale=-1,xscale=1]
%uncomment if require: \path (0,300); %set diagram left start at 0, and has height of 300

%Shape: Polygon Curved [id:ds9079456360528385] 
\draw   (52,77) .. controls (72,67) and (162,57) .. (142,77) .. controls (122,97) and (122,107) .. (142,137) .. controls (162,167) and (72,167) .. (52,137) .. controls (32,107) and (32,87) .. (52,77) -- cycle ;
%Shape: Polygon Curved [id:ds9931025453530924] 
\draw   (189,23) .. controls (209,13) and (299,3) .. (279,23) .. controls (259,43) and (216.2,25.2) .. (236.2,55.2) .. controls (256.2,85.2) and (209,113) .. (189,83) .. controls (169,53) and (169,33) .. (189,23) -- cycle ;
%Shape: Polygon Curved [id:ds98969863279118] 
\draw   (317.2,46.2) .. controls (337.2,36.2) and (277,16) .. (257,36) .. controls (237,56) and (236.2,50.2) .. (256.2,80.2) .. controls (276.2,110.2) and (331.2,110.2) .. (311.2,80.2) .. controls (291.2,50.2) and (297.2,56.2) .. (317.2,46.2) -- cycle ;
%Shape: Polygon Curved [id:ds3993560047064726] 
\draw   (196,128) .. controls (216,118) and (251.2,91.2) .. (286,128) .. controls (320.8,164.8) and (311.2,138.2) .. (331.2,168.2) .. controls (351.2,198.2) and (216,218) .. (196,188) .. controls (176,158) and (176,138) .. (196,128) -- cycle ;
%Shape: Polygon Curved [id:ds4662727835448177] 
\draw   (253.2,150.2) .. controls (273.2,140.2) and (235.2,117.2) .. (270,154) .. controls (304.8,190.8) and (265.2,137.2) .. (285.2,167.2) .. controls (305.2,197.2) and (261.2,204.2) .. (241.2,174.2) .. controls (221.2,144.2) and (233.2,160.2) .. (253.2,150.2) -- cycle ;

% Text Node
\draw (129,237.9) node [anchor=north west][inner sep=0.75pt]   [align=left] {A y C son conexos \\B es disconexo};
% Text Node
\draw (122,140.1) node [anchor=north west][inner sep=0.75pt]    {$A$};
% Text Node
\draw (286,5.1) node [anchor=north west][inner sep=0.75pt]    {$B$};


\end{tikzpicture}
\end{center}
Un subconjunto $H$ del espacio métrico $M$ es \textbf{disconexo}, si existen abiertos $A$ y $B\ni A\cap H\neq \emptyset, B\cap H\neq \emptyset, (A\cap H)\cap (B\cap H)=\emptyset$ y $(A\cap H)\cup (B\cup H)=H$

\begin{center}
    

\tikzset{every picture/.style={line width=0.75pt}} %set default line width to 0.75pt        

\begin{tikzpicture}[x=0.75pt,y=0.75pt,yscale=-1,xscale=1]
%uncomment if require: \path (0,300); %set diagram left start at 0, and has height of 300

%Shape: Polygon Curved [id:ds9931025453530924] 
\draw   (72.9,106.24) .. controls (97.83,94.25) and (175.13,76.22) .. (150.2,100.2) .. controls (125.27,124.18) and (106.81,108.87) .. (131.74,144.84) .. controls (156.67,180.81) and (97.83,214.14) .. (72.9,178.17) .. controls (47.97,142.2) and (47.97,118.22) .. (72.9,106.24) -- cycle ;
%Shape: Polygon Curved [id:ds98969863279118] 
\draw   (273.2,143.46) .. controls (298.13,131.47) and (223.08,107.25) .. (198.15,131.23) .. controls (173.22,155.21) and (172.22,148.26) .. (197.15,184.22) .. controls (222.08,220.19) and (290.65,220.19) .. (265.72,184.22) .. controls (240.78,148.26) and (248.26,155.45) .. (273.2,143.46) -- cycle ;
%Shape: Ellipse [id:dp5591846043661541] 
\draw  [color={rgb, 255:red, 255; green, 0; blue, 0 }  ,draw opacity=1 ][dash pattern={on 4.5pt off 4.5pt}] (36.46,113.52) .. controls (53.22,68.49) and (95.29,42.58) .. (130.44,55.66) .. controls (165.59,68.74) and (180.49,115.85) .. (163.74,160.88) .. controls (146.98,205.91) and (104.91,231.82) .. (69.76,218.74) .. controls (34.61,205.66) and (19.71,158.55) .. (36.46,113.52) -- cycle ;
%Shape: Ellipse [id:dp10167958292516732] 
\draw  [color={rgb, 255:red, 255; green, 0; blue, 0 }  ,draw opacity=1 ][dash pattern={on 4.5pt off 4.5pt}] (174.46,141.52) .. controls (191.22,96.49) and (233.29,70.58) .. (268.44,83.66) .. controls (303.59,96.74) and (318.49,143.85) .. (301.74,188.88) .. controls (284.98,233.91) and (242.91,259.82) .. (207.76,246.74) .. controls (172.61,233.66) and (157.71,186.55) .. (174.46,141.52) -- cycle ;

% Text Node
\draw (172.97,35.53) node [anchor=north west][inner sep=0.75pt]    {$H$};
% Text Node
\draw (68,31.1) node [anchor=north west][inner sep=0.75pt]    {$A$};
% Text Node
\draw (269,57.1) node [anchor=north west][inner sep=0.75pt]    {$B$};


\end{tikzpicture}
\end{center}
\end{definition}
\section{4 de febrero de 2021}

\begin{definition}
\begin{center}
    

\tikzset{every picture/.style={line width=0.75pt}} %set default line width to 0.75pt        

\begin{tikzpicture}[x=0.75pt,y=0.75pt,yscale=-1,xscale=1]
%uncomment if require: \path (0,300); %set diagram left start at 0, and has height of 300

%Straight Lines [id:da5220143812109725] 
\draw [color={rgb, 255:red, 255; green, 0; blue, 0 }  ,draw opacity=1 ]   (122,146) -- (122.19,176.2) ;
\draw [shift={(122.2,178.2)}, rotate = 269.64] [color={rgb, 255:red, 255; green, 0; blue, 0 }  ,draw opacity=1 ][line width=0.75]    (10.93,-3.29) .. controls (6.95,-1.4) and (3.31,-0.3) .. (0,0) .. controls (3.31,0.3) and (6.95,1.4) .. (10.93,3.29)   ;
%Straight Lines [id:da9488864909412285] 
\draw [color={rgb, 255:red, 255; green, 0; blue, 0 }  ,draw opacity=1 ]   (119,72) -- (120.09,52.2) ;
\draw [shift={(120.2,50.2)}, rotate = 453.15] [color={rgb, 255:red, 255; green, 0; blue, 0 }  ,draw opacity=1 ][line width=0.75]    (10.93,-3.29) .. controls (6.95,-1.4) and (3.31,-0.3) .. (0,0) .. controls (3.31,0.3) and (6.95,1.4) .. (10.93,3.29)   ;
%Straight Lines [id:da16251395696621607] 
\draw [color={rgb, 255:red, 255; green, 0; blue, 0 }  ,draw opacity=1 ]   (164,94) -- (216.27,107.69) ;
\draw [shift={(218.2,108.2)}, rotate = 194.68] [color={rgb, 255:red, 255; green, 0; blue, 0 }  ,draw opacity=1 ][line width=0.75]    (10.93,-3.29) .. controls (6.95,-1.4) and (3.31,-0.3) .. (0,0) .. controls (3.31,0.3) and (6.95,1.4) .. (10.93,3.29)   ;
%Straight Lines [id:da03666972810090552] 
\draw [color={rgb, 255:red, 255; green, 0; blue, 0 }  ,draw opacity=1 ]   (177,131) -- (212.41,113.1) ;
\draw [shift={(214.2,112.2)}, rotate = 513.19] [color={rgb, 255:red, 255; green, 0; blue, 0 }  ,draw opacity=1 ][line width=0.75]    (10.93,-3.29) .. controls (6.95,-1.4) and (3.31,-0.3) .. (0,0) .. controls (3.31,0.3) and (6.95,1.4) .. (10.93,3.29)   ;
%Straight Lines [id:da5243518205926371] 
\draw [color={rgb, 255:red, 255; green, 0; blue, 0 }  ,draw opacity=1 ]   (165,42) -- (199.2,42.19) ;
\draw [shift={(201.2,42.2)}, rotate = 180.32] [color={rgb, 255:red, 255; green, 0; blue, 0 }  ,draw opacity=1 ][line width=0.75]    (10.93,-3.29) .. controls (6.95,-1.4) and (3.31,-0.3) .. (0,0) .. controls (3.31,0.3) and (6.95,1.4) .. (10.93,3.29)   ;

% Text Node
\draw (93,32.9) node [anchor=north west][inner sep=0.75pt]   [align=left] {Conexo};
% Text Node
\draw (84,82.9) node [anchor=north west][inner sep=0.75pt]   [align=left] {Conexidad};
% Text Node
\draw (84,123.9) node [anchor=north west][inner sep=0.75pt]   [align=left] {Compacidad};
% Text Node
\draw (84,175.9) node [anchor=north west][inner sep=0.75pt]   [align=left] {Compacto};
% Text Node
\draw (106,202.1) node [anchor=north west][inner sep=0.75pt]  [color={rgb, 255:red, 65; green, 117; blue, 5 }  ,opacity=1 ]  {$abierto$};
% Text Node
\draw (111,6.1) node [anchor=north west][inner sep=0.75pt]  [color={rgb, 255:red, 65; green, 117; blue, 5 }  ,opacity=1 ]  {$abierto$};
% Text Node
\draw (230,92.9) node [anchor=north west][inner sep=0.75pt]   [align=left] {Preserva baja aplicación \\de funciones continuas};
% Text Node
\draw (212,31.9) node [anchor=north west][inner sep=0.75pt]   [align=left] {Valor intermedio};


\end{tikzpicture}
\end{center}
\end{definition}

\begin{remark}
$A$ es conexo, si $A$ no es disconexo. \newline 
\textbf{Disconexo:}
\begin{center}
    

\tikzset{every picture/.style={line width=0.75pt}} %set default line width to 0.75pt        

\begin{tikzpicture}[x=0.75pt,y=0.75pt,yscale=-1,xscale=1]
%uncomment if require: \path (0,300); %set diagram left start at 0, and has height of 300

%Shape: Circle [id:dp8455402914572695] 
\draw  [dash pattern={on 4.5pt off 4.5pt}] (100,94.6) .. controls (100,60.14) and (127.94,32.2) .. (162.4,32.2) .. controls (196.86,32.2) and (224.8,60.14) .. (224.8,94.6) .. controls (224.8,129.06) and (196.86,157) .. (162.4,157) .. controls (127.94,157) and (100,129.06) .. (100,94.6) -- cycle ;
%Shape: Circle [id:dp43487142688104174] 
\draw  [dash pattern={on 4.5pt off 4.5pt}] (241,103.6) .. controls (241,69.14) and (268.94,41.2) .. (303.4,41.2) .. controls (337.86,41.2) and (365.8,69.14) .. (365.8,103.6) .. controls (365.8,138.06) and (337.86,166) .. (303.4,166) .. controls (268.94,166) and (241,138.06) .. (241,103.6) -- cycle ;
%Shape: Ellipse [id:dp9061110877616693] 
\draw   (134.52,115.1) .. controls (126.36,102.38) and (132.93,83.61) .. (149.2,73.17) .. controls (165.46,62.73) and (185.26,64.57) .. (193.43,77.28) .. controls (201.59,89.99) and (195.02,108.76) .. (178.75,119.21) .. controls (162.48,129.65) and (142.68,127.81) .. (134.52,115.1) -- cycle ;
%Shape: Ellipse [id:dp419797908656945] 
\draw   (273.95,122.51) .. controls (265.79,109.8) and (272.36,91.03) .. (288.62,80.58) .. controls (304.89,70.14) and (324.69,71.98) .. (332.85,84.69) .. controls (341.01,97.4) and (334.44,116.17) .. (318.18,126.62) .. controls (301.91,137.06) and (282.11,135.22) .. (273.95,122.51) -- cycle ;
%Straight Lines [id:da9933913068723642] 
\draw    (185,33) -- (229.2,22.2) ;
%Straight Lines [id:da26845766244024405] 
\draw    (265,28) -- (296.2,38.2) ;

% Text Node
\draw (353,39.1) node [anchor=north west][inner sep=0.75pt]    {$H$};
% Text Node
\draw (92,148.1) node [anchor=north west][inner sep=0.75pt]    {$G$};
% Text Node
\draw (242,18.1) node [anchor=north west][inner sep=0.75pt]    {$B$};


\end{tikzpicture}
\end{center}

Si existen $G$ y $H$ (Disconexión de $B$) $\ni$
\begin{enumerate}
    \item $G\cap B\neq \emptyset$ y $H\cap B\neq \emptyset$
    \item $(G\cap B)\cap (H\cap B)=\emptyset$
    \item $(G\cap B)\cup (H\cap B)=B$
\end{enumerate}
\end{remark}

\begin{example}
\begin{enumerate}
    \item $\mathbb{Z}$ es disconexo en $\mathbb{R}$. En efecto, considere: $$G=(-\infty,\frac{1}{2})\qquad H=(\frac{1}{2},\infty)$$
    $\implies G$ y $H$ son una disconexión de $\mathbb{Z}\subseteq \mathbb{R}$
    \item $\mathbb{Q}$ es disconexo en $\mathbb{R}$. En efecto, sea la disconexión: $$G=(-\infty,\pi)\qquad H=(\pi,\infty)$$
\end{enumerate}
\end{example}

\begin{theorem}
$I=[0,1]$ es conexo en $\mathbb{R}$.
\end{theorem}
\begin{proof}
Supóngase por el absurdo que $A$ y $B$ son una disconexión de $I$; i.e., $A\cap I$ y $B\cap I$ son no vacíos, disjuntos y su unión es $I$.
\begin{center}
    



\tikzset{every picture/.style={line width=0.75pt}} %set default line width to 0.75pt        

\begin{tikzpicture}[x=0.75pt,y=0.75pt,yscale=-1,xscale=1]
%uncomment if require: \path (0,300); %set diagram left start at 0, and has height of 300

%Straight Lines [id:da035022553384921884] 
\draw    (83,139) -- (193.2,139.2) (93.01,135.02) -- (92.99,143.02)(103.01,135.04) -- (102.99,143.04)(113.01,135.05) -- (112.99,143.05)(123.01,135.07) -- (122.99,143.07)(133.01,135.09) -- (132.99,143.09)(143.01,135.11) -- (142.99,143.11)(153.01,135.13) -- (152.99,143.13)(163.01,135.15) -- (162.99,143.15)(173.01,135.16) -- (172.99,143.16)(183.01,135.18) -- (182.99,143.18)(193.01,135.2) -- (192.99,143.2) ;
%Shape: Circle [id:dp5235472636962212] 
\draw  [fill={rgb, 255:red, 255; green, 9; blue, 9 }  ,fill opacity=1 ] (138.1,139.1) .. controls (138.1,134.9) and (141.5,131.5) .. (145.7,131.5) .. controls (149.9,131.5) and (153.3,134.9) .. (153.3,139.1) .. controls (153.3,143.3) and (149.9,146.7) .. (145.7,146.7) .. controls (141.5,146.7) and (138.1,143.3) .. (138.1,139.1) -- cycle ;

% Text Node
\draw (101,126.9) node [anchor=north west][inner sep=0.75pt]  [font=\Large] [align=left] {[};
% Text Node
\draw (185.07,152.07) node [anchor=north west][inner sep=0.75pt]  [font=\Large,rotate=-179.7] [align=left] {[};
% Text Node
\draw (54,120.1) node [anchor=north west][inner sep=0.75pt]  [font=\Large,color={rgb, 255:red, 74; green, 144; blue, 226 }  ,opacity=1 ]  {$( \ \ \ \ \ \ \ \ \ \ \ \ )$};
% Text Node
\draw (144,121.1) node [anchor=north west][inner sep=0.75pt]  [font=\Large,color={rgb, 255:red, 208; green, 2; blue, 27 }  ,opacity=1 ]  {$( \ \ \ \ \ \ \ \ \ \ \ \ )$};
% Text Node
\draw (92,104.1) node [anchor=north west][inner sep=0.75pt]    {$A$};
% Text Node
\draw (187,105.1) node [anchor=north west][inner sep=0.75pt]    {$B$};
% Text Node
\draw (97,154.1) node [anchor=north west][inner sep=0.75pt]    {$0$};
% Text Node
\draw (177,157.1) node [anchor=north west][inner sep=0.75pt]    {$1$};
% Text Node
\draw (141,106.1) node [anchor=north west][inner sep=0.75pt]    {$c$};


\end{tikzpicture}
\end{center}
\begin{itemize}
    \item Suponga que $1\in B$. Como $I$ es acotado. $\implies A\cap I$ y $B\cap I$ también son acotados. Entonces, por el principio del supremo, $\exists c=sup(A\cap I)>0$ y $c\in A\cup B$
    \item Si $c\in A\implies c<1$\newline 
    $\implies$ como $A$ es abierto $\implies \exists B_r(c)\subset A\implies \exists \alpha \in A\ni c<\alpha (\to\gets)\implies c\not\in A$
    \item Si $c\in B\implies$ como $B$ es abierto. $\implies \exists c_1\in B\ni c_1<c$ y es tal que $[c_1,c]\cap (A\cap i)=\emptyset$ (i.e. $c_1$ es una cota superior de $A\cap I$ y es menor que $c$)($\to\gets$). Entonces, que $c\not\in B(\to\gets)$. 
\end{itemize}
$\implies [0,1]$ es conexo.
\end{proof}

\begin{corollary}
$(0,1)$ es conexo. \marginnote{Si $x$ es un intervalo $\implies x$ es conexo.}
\end{corollary}


\begin{theorem}
$\mathbb{R}^n$ es conexo. 
\end{theorem}

\begin{proof}
Supóngase, por el absurdo, que $A$ y $B$ son una disconexión de $\mathbb{R}^n$
\begin{center}
    

\tikzset{every picture/.style={line width=0.75pt}} %set default line width to 0.75pt        

\begin{tikzpicture}[x=0.75pt,y=0.75pt,yscale=-1,xscale=1]
%uncomment if require: \path (0,300); %set diagram left start at 0, and has height of 300

%Shape: Rectangle [id:dp5961020233676221] 
\draw   (50.2,68.2) -- (235.2,68.2) -- (235.2,169) -- (50.2,169) -- cycle ;
%Curve Lines [id:da049078825361805634] 
\draw    (128,72) .. controls (168,42) and (126.2,196.2) .. (166.2,166.2) ;
%Straight Lines [id:da05214217205895033] 
\draw  [dash pattern={on 0.84pt off 2.51pt}]  (94,127) -- (182.2,112.2) ;

% Text Node
\draw (246,59.1) node [anchor=north west][inner sep=0.75pt]    {$\mathbb{R}$};
% Text Node
\draw (62,82.1) node [anchor=north west][inner sep=0.75pt]    {$A$};
% Text Node
\draw (207,81.1) node [anchor=north west][inner sep=0.75pt]    {$B$};
% Text Node
\draw (92,130.1) node [anchor=north west][inner sep=0.75pt]  [color={rgb, 255:red, 65; green, 117; blue, 5 }  ,opacity=1 ]  {$x$};
% Text Node
\draw (184,113.1) node [anchor=north west][inner sep=0.75pt]  [color={rgb, 255:red, 65; green, 117; blue, 5 }  ,opacity=1 ]  {$y$};


\end{tikzpicture}
\end{center}

Sean $x\in A$ y $y\in B$, y considere el segmento de recta que une $x$ con $y$:
$$S=\{(1-t)x+ty:t\in[0,1]\}$$
Sean: $$A_1=\{t\in \mathbb{R}\ni (1-t)x+ty\in A\}$$
$$B_1=\{t\in\mathbb{R}\ni(1-t)x+ty\in B\}$$
\marginnote{$[0,1]$}
$\implies A_1\cap B_1=\emptyset (\to\gets)$, ya que $A_1,B_1$ serían una disconexión de $[0,1]$. Entonces, $\mathbb{R}^n$ es convexo. 
\end{proof}

\begin{remark}
\begin{center}
    

\tikzset{every picture/.style={line width=0.75pt}} %set default line width to 0.75pt        

\begin{tikzpicture}[x=0.75pt,y=0.75pt,yscale=-1,xscale=1]
%uncomment if require: \path (0,300); %set diagram left start at 0, and has height of 300

%Shape: Ellipse [id:dp658140374880671] 
\draw   (98.09,156.37) .. controls (79.87,142.76) and (74.48,119.17) .. (86.05,103.69) .. controls (97.62,88.2) and (121.77,86.68) .. (139.98,100.29) .. controls (158.2,113.9) and (163.59,137.48) .. (152.02,152.97) .. controls (140.45,168.46) and (116.31,169.98) .. (98.09,156.37) -- cycle ;
%Shape: Polygon Curved [id:ds3405091393911408] 
\draw   (207,98) .. controls (227,88) and (369.8,75.8) .. (297,98) .. controls (224.2,120.2) and (227.2,121.2) .. (297,158) .. controls (366.8,194.8) and (227,188) .. (207,158) .. controls (187,128) and (187,108) .. (207,98) -- cycle ;

% Text Node
\draw (79,63.9) node [anchor=north west][inner sep=0.75pt]   [align=left] {Conexo};
% Text Node
\draw (215,63.9) node [anchor=north west][inner sep=0.75pt]   [align=left] {Convexo};


\end{tikzpicture}
\end{center}
\end{remark}

\begin{theorem}
Los únicos conjuntos abiertos y cerrados de $\mathbb{R}^n$ son $\emptyset$ y $\mathbb{R}^n$
\end{theorem}
\begin{proof}
Supóngase, por el absurdo, que $A\subset \mathbb{R}^n$, $A\neq \emptyset$ y $A\neq \mathbb{R}^n$, es abierto y cerrado de $\mathbb{R}^n$. Como $A$ es cerrado $\implies A^c=B$ es abierto. 
$\implies A\neq \emptyset \implies B\neq \emptyset,A\cap B=\emptyset$ y $A\cup B=\mathbb{R}^n$. $\implies A$ y $B$ forman una disconexión de $\mathbb{R}^n(\to\gets)$. $\implies$ los únicos abiertos y cerrados de $\mathbb{R}^n$ son $\emptyset$ y $\mathbb{R}^n$ 
\end{proof}

\begin{theorem}
Un subconjunto de $\mathbb{R}$ es conexo ssi es un intervalo. 
\end{theorem}
\begin{theorem}
\begin{itemize}
    \item $(\gets)$ A probar cada intervalo de $\mathbb{R}$ es un conexo (ver prueba de: $[0,1]$ es conexo=.
    \item $(\to)$ Sea $C\subset \mathbb{R}$, conexo. A probar: $C$ es un intervalo. Sean $a,b\in C\ni a<b$ y sea $x\in\mathbb{R}\ni a<x<v$. A probar: $x\in C$\begin{center}
        

\tikzset{every picture/.style={line width=0.75pt}} %set default line width to 0.75pt        

\begin{tikzpicture}[x=0.75pt,y=0.75pt,yscale=-1,xscale=1]
%uncomment if require: \path (0,300); %set diagram left start at 0, and has height of 300

%Straight Lines [id:da6796265049147991] 
\draw    (89,141) -- (191.2,141.2) ;
%Straight Lines [id:da07800145146886805] 
\draw    (100.2,132.2) -- (100.2,152.2) ;
%Straight Lines [id:da617799668252107] 
\draw    (171,132) -- (171.2,150.2) ;
%Shape: Circle [id:dp7757914759860494] 
\draw  [color={rgb, 255:red, 126; green, 211; blue, 33 }  ,draw opacity=1 ][fill={rgb, 255:red, 65; green, 117; blue, 5 }  ,fill opacity=1 ] (125.9,141.1) .. controls (125.9,137.18) and (129.08,134) .. (133,134) .. controls (136.92,134) and (140.1,137.18) .. (140.1,141.1) .. controls (140.1,145.02) and (136.92,148.2) .. (133,148.2) .. controls (129.08,148.2) and (125.9,145.02) .. (125.9,141.1) -- cycle ;

% Text Node
\draw (94,153.1) node [anchor=north west][inner sep=0.75pt]    {$a$};
% Text Node
\draw (163,157.1) node [anchor=north west][inner sep=0.75pt]    {$b$};


\end{tikzpicture}
    \end{center}
    Si $x\not\in C\implies (-\infty,c)$ y $(x,\infty)$ formar una disconexión de $c$. 
\end{itemize}
\end{theorem}
\section{8 de febrero de 2021}

\subsection{Compactos}
\begin{definition}
Sea $A$ un subconjunto del espacio métrico M. Decimos que la familia de abiertos $\{G_1\}_{i\in I}$ de $M$ es una cubierta de $A$, si
$$A\subset \bigcup_{i\in I}G_i $$
\end{definition}
\begin{remark}
En el caso de $M$, la cubierta abierta debe cumplir: $M=\bigcup_{i\in I}G_i$
\end{remark}

\begin{definition}
Un subconjunto $A$ del espacio métrico $M$ es \textbf{compacto} si cada abierta de $A$ tiene subcubierta finita \marginnote{Sigue cubriendo al conjunto $A$}. 
\end{definition}

\begin{example}
Sea $k=\{x_1,...,x_n\}$ un subconjunto finito de $\mathbb{R}^n$ y sea $G=\{G_i\}_{i\in I}$ una cubierta abierta de $k$ (i.e. $\bigcup_{i\in I}G_i\supset k$. Dado que $k$ es finito, basta un número finito de los $G_i$ para cubrir a $k\implies k$ es compacto. 
\end{example}
\marginnote{Cerrado y acotado en $\mathbb{R}$ es compacto. En otro caso es necesario investigar}
\begin{example}
Sea $H=[0,\infty)\subseteq \mathbb{R}$ no es compacto. En efecto, sea $G_n =(-1,n), n\in\mathbb{Z}^+$.\newline\newline 
$\implies G=\{G_n\}$ es una cubierta abierta de $H$. Suponga que $\{G_{n_1},G_{n_2},...,G_{n_k}\}$ es una subcolección de $G$. Sea $M=max\{n_1,n_2,...,n_k\}$. Entonces, $G_{n_i}\subseteq G_M, i=1,...,k\implies G_M=\bigcup_{i=1}^k G_{n_i}$, pero en particular, $M\not\in \bigcup_{i=1}^k G_{n_i}\implies \{G_{n_1},...,G_{n_k}\}$ no cubre a $H$. Entonces, $G$ no tiene cubierta finita para $H$. Implica, $H$ no es compacto. 
\end{example}

\begin{example}
Sea $H=(0,1)\subseteq \mathbb{R}$ y considere:
$$G_n(\frac{1}{n},1-\frac{1}{n}), \quad n>2$$

$\implies G=\{G_n\}$ es una cubierta abierta de $H$, pero $G$ no tiene subcubierta finita para $H\implies H$ no es compacta. 
\end{example}

\begin{proposition}
Sea $F$ un subconjunto cerrado de un espacio métrica compacto $M$. Entonces, $F$ es compacto. 
\begin{center}
    

\tikzset{every picture/.style={line width=0.75pt}} %set default line width to 0.75pt        

\begin{tikzpicture}[x=0.75pt,y=0.75pt,yscale=-1,xscale=1]
%uncomment if require: \path (0,300); %set diagram left start at 0, and has height of 300

%Shape: Rectangle [id:dp7494793647468762] 
\draw   (24.2,66.2) -- (170,66.2) -- (170,147) -- (24.2,147) -- cycle ;
%Shape: Polygon Curved [id:ds921677921975807] 
\draw   (90.2,82.2) .. controls (110.2,72.2) and (127.2,63.2) .. (107.2,83.2) .. controls (87.2,103.2) and (126.2,71.2) .. (146.2,101.2) .. controls (166.2,131.2) and (109.2,154.2) .. (89.2,124.2) .. controls (69.2,94.2) and (70.2,92.2) .. (90.2,82.2) -- cycle ;

% Text Node
\draw (187,64.1) node [anchor=north west][inner sep=0.75pt]    {$M$};
% Text Node
\draw (105,92.1) node [anchor=north west][inner sep=0.75pt]    {$F$};


\end{tikzpicture}
\end{center}
\end{proposition}

\begin{proof}
Sea $G=\{G_i\}$ una cubierta abierta de F. Como $F^c$ es abierto $\implies (\bigcup_{i\in I}G_i)\cup F^c$ es cubierta abierta de $M$ (i.e $(\bigcup_{i=I}G_i)\cup F^c=M$). Entonces, como $M$ es compacto, existe una subcubierta finita de $M, \{G_{i_1},G_{i_2},..., G_{i_n}, F^c\}$, tal que: 
$$G_{i_1}\cup G_{i_2}\cup... \cup G_{i_n}\cup F^c=M$$
$\implies \{ G_{i_1},...,G_{i_n}$ es una subcubierta finita para $F\implies$ F es compacto. 
\end{proof}

\begin{theorem}[Heine-Borel]
Un subconjunto $S$ de $\mathbb{R}^n$ es compacto ssi es cerrado y acotado. 
\end{theorem}

\begin{example}
\begin{itemize}
    \item (0,1) no es compacto, ya que no es cerrado. 
    \item [0,1] es compacto, por Heine-Borel. 
\end{itemize}
\end{example}

\begin{remark}
\begin{enumerate}
    \item Si $S\subseteq \mathbb{R}$, compacto, $\implies S$ es cerrado y acotado. 
    \item Si $S\mathbb{R}$ es cerrado y acotado $\implies S$ es secuencialmente compacto $\implies  S$ es compacto $\implies S$ es compacto. 
\end{enumerate}
\end{remark}
\marginnote{Teorema de Tíkonov (Tychonoff)- Producto de una colección cualquiera de conjuntos compactos es compacto.}

\begin{remark}
Un espacio métrico $M$ es de Lindelöf si cada cubierta abierta de $M$ tiene una subcubierta contable. 
Un subconjunto $A\subseteq M$ es un  
\end{remark}

\begin{theorem}
Si $S\subseteq\mathbb{R}$ es compacto, entonces $S$ es cerrado y acotado. 
\end{theorem}
\begin{proof}
A probar: $S$ es acotado,\newline 
Considere, para $m\in \mathbb{Z}^+$, $H_m=(-m,m)$. Como cada $H_m$ es abierto  y $ S\subset\bigcup_{i=1}^{\infty} H_m=\mathbb{R} \implies \{H_m:m\in\mathbb{Z}^+\}$ es cubierta abierta de $S$. Como $S$ es compacto $\implies$ Hay una subcubierta finita $\{H_{m_1},...,H_{m_n}\}$ para $S$, i.e $S\subseteq \bigcup_{i=1}^n H_{m_n})=H_m=(-M,M)\implies M$ es acotado.\newline\newline 

A probar: $S$ es cerrado. $\leftrightarrow S^c$ es abierto. 

\begin{center}
    

\tikzset{every picture/.style={line width=0.75pt}} %set default line width to 0.75pt        

\begin{tikzpicture}[x=0.75pt,y=0.75pt,yscale=-1,xscale=1]
%uncomment if require: \path (0,300); %set diagram left start at 0, and has height of 300

%Shape: Rectangle [id:dp7494793647468762] 
\draw   (24.2,66.2) -- (170,66.2) -- (170,147) -- (24.2,147) -- cycle ;
%Shape: Polygon Curved [id:ds921677921975807] 
\draw   (44.2,86.2) .. controls (64.2,76.2) and (57.2,75.2) .. (71.2,90.2) .. controls (85.2,105.2) and (48.2,101.2) .. (68.2,131.2) .. controls (88.2,161.2) and (24.2,96.2) .. (44.2,86.2) -- cycle ;
%Shape: Circle [id:dp7359910387512897] 
\draw  [dash pattern={on 0.84pt off 2.51pt}] (103,114) .. controls (103,100.19) and (114.19,89) .. (128,89) .. controls (141.81,89) and (153,100.19) .. (153,114) .. controls (153,127.81) and (141.81,139) .. (128,139) .. controls (114.19,139) and (103,127.81) .. (103,114) -- cycle ;

% Text Node
\draw (187,63.1) node [anchor=north west][inner sep=0.75pt]    {$M$};
% Text Node
\draw (55,89.1) node [anchor=north west][inner sep=0.75pt]    {$S$};
% Text Node
\draw (142,71.1) node [anchor=north west][inner sep=0.75pt]    {$S^{c}$};
% Text Node
\draw (120,104.1) node [anchor=north west][inner sep=0.75pt]    {$u$};


\end{tikzpicture}
\end{center}
Sea $u\in S^c$ y considere: 
$$G_n=\{y\in \mathbb{R}:|y-u|> 1/n\}, n\in\mathbb{Z}^+$$
\begin{center}
    

\tikzset{every picture/.style={line width=0.75pt}} %set default line width to 0.75pt        

\begin{tikzpicture}[x=0.75pt,y=0.75pt,yscale=-1,xscale=1]
%uncomment if require: \path (0,300); %set diagram left start at 0, and has height of 300

%Shape: Rectangle [id:dp7494793647468762] 
\draw   (24.2,67.11) -- (223.49,67.11) -- (223.49,218.2) -- (24.2,218.2) -- cycle ;
%Shape: Rectangle [id:dp01772513124623576] 
\draw   (65.21,97.03) -- (190.96,97.03) -- (190.96,190.53) -- (65.21,190.53) -- cycle ;
%Shape: Circle [id:dp6111024653194649] 
\draw  [dash pattern={on 0.84pt off 2.51pt}] (100,145.6) .. controls (100,125.94) and (115.94,110) .. (135.6,110) .. controls (155.26,110) and (171.2,125.94) .. (171.2,145.6) .. controls (171.2,165.26) and (155.26,181.2) .. (135.6,181.2) .. controls (115.94,181.2) and (100,165.26) .. (100,145.6) -- cycle ;

% Text Node
\draw (234.26,73.8) node [anchor=north west][inner sep=0.75pt]    {$\mathbb{R}$};
% Text Node
\draw (197,84.1) node [anchor=north west][inner sep=0.75pt]    {$S^{c}$};
% Text Node
\draw (128,138.1) node [anchor=north west][inner sep=0.75pt]    {$u$};


\end{tikzpicture}
\end{center}
Note que los $G_n$ son abiertos y $\bigcup_{n=1}^\infty G_n=\mathbb{R}-\{u\}$. Como $u\not\in S\implies S\subseteq \bigcup_{n=1}^\infty G_n$. $\implies \{G_n\}$ es una cubierta abierta d $S$. Como $S$ es compacto $\implies \exists m\in\mathbb{Z}^+\ni S\subseteq\bigcup_{n=1}^m G_n=G_m\implies S\cap (u-1/m,u+1/m)=\emptyset\implies (u-1/m,u+1/m)\subset S^c\implies$ es abierto $\implies S$ es cerrado. 
\end{proof}

\begin{remark}
¿Por qué Heine-Borel no aplica en un espacio métrico cualquiera? ¿Se cumple alguna de las implicaciones?
\end{remark}

\pagelayout{wide} % No margins
\addpart{Cosas interesantes...}
\pagelayout{margin} % Restore margins

\appendix % From here onwards, chapters are numbered with letters, as is the appendix convention


%----------------------------------------------------------------------------------------

\backmatter % Denotes the end of the main document content
\setchapterstyle{plain} % Output plain chapters from this point onwards

%----------------------------------------------------------------------------------------
%	BIBLIOGRAPHY
%----------------------------------------------------------------------------------------

% The bibliography needs to be compiled with biber using your LaTeX editor, or on the command line with 'biber main' from the template directory

\defbibnote{bibnote}{Referencias utilizadas en las notas.\par\bigskip} % Prepend this text to the bibliography
\printbibliography[heading=bibintoc, title=Bibliography, prenote=bibnote] % Add the bibliography heading to the ToC, set the title of the bibliography and output the bibliography note

%----------------------------------------------------------------------------------------
%	NOMENCLATURE
%----------------------------------------------------------------------------------------

% The nomenclature needs to be compiled on the command line with 'makeindex main.nlo -s nomencl.ist -o main.nls' from the template directory



% Prepend this text to the nomenclature

\printnomenclature % Output the nomenclature

%----------------------------------------------------------------------------------------
%	GREEK ALPHABET
% 	Originally from https://gitlab.com/jim.hefferon/linear-algebra
%----------------------------------------------------------------------------------------

\vspace{1cm}

{\usekomafont{chapter}Letras Griegas con pronunciación} \\[2ex]
\begin{center}
	\newcommand{\pronounced}[1]{\hspace*{.2em}\small\textit{#1}}
	\begin{tabular}{l l @{\hspace*{3em}} l l}
		\toprule
		Character & Name & Character & Name \\ 
		\midrule
		$\alpha$ & alpha \pronounced{AL-fuh} & $\nu$ & nu \pronounced{NEW} \\
		$\beta$ & beta \pronounced{BAY-tuh} & $\xi$, $\Xi$ & xi \pronounced{KSIGH} \\ 
		$\gamma$, $\Gamma$ & gamma \pronounced{GAM-muh} & o & omicron \pronounced{OM-uh-CRON} \\
		$\delta$, $\Delta$ & delta \pronounced{DEL-tuh} & $\pi$, $\Pi$ & pi \pronounced{PIE} \\
		$\epsilon$ & epsilon \pronounced{EP-suh-lon} & $\rho$ & rho \pronounced{ROW} \\
		$\zeta$ & zeta \pronounced{ZAY-tuh} & $\sigma$, $\Sigma$ & sigma \pronounced{SIG-muh} \\
		$\eta$ & eta \pronounced{AY-tuh} & $\tau$ & tau \pronounced{TOW (as in cow)} \\
		$\theta$, $\Theta$ & theta \pronounced{THAY-tuh} & $\upsilon$, $\Upsilon$ & upsilon \pronounced{OOP-suh-LON} \\
		$\iota$ & iota \pronounced{eye-OH-tuh} & $\phi$, $\Phi$ & phi \pronounced{FEE, or FI (as in hi)} \\
		$\kappa$ & kappa \pronounced{KAP-uh} & $\chi$ & chi \pronounced{KI (as in hi)} \\
		$\lambda$, $\Lambda$ & lambda \pronounced{LAM-duh} & $\psi$, $\Psi$ & psi \pronounced{SIGH, or PSIGH} \\
		$\mu$ & mu \pronounced{MEW} & $\omega$, $\Omega$ & omega \pronounced{oh-MAY-guh} \\
		\bottomrule
	\end{tabular} \\[1.5ex]
\end{center}

%----------------------------------------------------------------------------------------
%	GLOSSARY
%----------------------------------------------------------------------------------------

% The glossary needs to be compiled on the command line with 'makeglossaries main' from the template directory

\newglossaryentry{computer}{
	name=computer,
	description={is a programmable machine that receives input, stores and manipulates data, and provides output in a useful format}
}

% Glossary entries (used in text with e.g. \acrfull{fpsLabel} or \acrshort{fpsLabel})
\newacronym[longplural={Frames per Second}]{fpsLabel}{FPS}{Frame per Second}
\newacronym[longplural={Tables of Contents}]{tocLabel}{TOC}{Table of Contents}

\setglossarystyle{listgroup} % Set the style of the glossary (see https://en.wikibooks.org/wiki/LaTeX/Glossary for a reference)
\printglossary[title=Special Terms, toctitle=List of Terms] 
\printindex % Output the index



\end{document}
