\documentclass[a4paper,12pt]{article}
\usepackage[top = 2.5cm, bottom = 2.5cm, left = 2.5cm, right = 2.5cm]{geometry}
% Unfortunately, LaTeX has a hard time interpreting German Umlaute. The following two lines and packages should help. If it doesn't work for you please let me know.
\usepackage[T1]{fontenc}
\usepackage[utf8]{inputenc}
% The following two packages - multirow and booktabs - are needed to create nice looking tables.
\usepackage{multirow} % Multirow is for tables with multiple rows within one cell.
\usepackage{booktabs} % For even nicer tables.
% As we usually want to include some plots (.pdf files) we need a package for that.
\usepackage{graphicx}
% The default setting of LaTeX is to indent new paragraphs. This is useful for articles. But not really nice for homework problem sets. The following command sets the indent to 0.
\usepackage[spanish]{babel}
\usepackage{setspace}
\setlength{\parindent}{0in}
% Package to place figures where you want them.
\usepackage{float}
% The fancyhdr package let's us create nice headers.
\usepackage{fancyhdr}
\usepackage{amsmath}
\usepackage{amssymb}
\usepackage{amsthm}
\usepackage{natbib}
\usepackage{graphicx}
\usepackage{subcaption}
\usepackage{booktabs}
\usepackage{etoolbox}
\usepackage{apalike}
\AtBeginEnvironment{align}{\setcounter{equation}{0}}

\pagestyle{fancy}

\fancyhf{}

\lhead{\footnotesize HT 1-Revisión}
\rhead{\footnotesize  Rompich}
\cfoot{\footnotesize \thepage}

\begin{document}
    \thispagestyle{empty} % This command disables the header on the first page.

    \begin{tabular}{p{15.5cm}} % This is a simple tabular environment to align your text nicely
    \begin{tabbing}
    Universidad del Valle de Guatemala \\
    Departamento de Matemática\\
    Licenciatura en Matemática Aplicada
    \\14 de febrero de 2021  \\
    Rudik Roberto Rompich   - Carné: 19857\\
    \end{tabbing}
    Análisis de Variable Real 1 - Dorval Carías \\
    \hline % \hline produces horizontal lines.
    \\
    \end{tabular} % Our tabular environment ends here.
    \vspace*{0.3cm} % Now we want to add some vertical space in between the line and our title.
    \begin{center} % Everything within the center environment is centered.
    {\Large \bf HT 1 - Revisión
} % <---- Don't forget to put in the right number
        \vspace{2mm}
    \end{center}
    \vspace{0.4cm}


1.\textbf{ Muestre que el campo $\mathbb{Q}$ de los racionales es arquimediano.} 
\begin{proof}
Por definición, un campo ordenado $\mathbb{F}$ tiene la propiedad arquimediana si $\forall x\in\mathbb{F}\exists n\in\mathbb{Z}^+\ni x<n$. Entonces, sea $\mathbb{Q}$ el campo de los racionales, es decir $\exists$ unos elementos $m$ y $n\ni \frac{n}{m}\in \mathbb{Q}$ en donde $ m\neq 0$ y $m,n\in \mathbb{Z}$. Entonces, se deben cumplir, según el Teorema 1.4.2 de \cite{abbott2012understanding}: 

\begin{equation}\label{eq:1}
x\in \mathbb{Q}, \exists n\in\mathbb{N} \text{ que satisfaga la desigualdad } n>x    
\end{equation}
\begin{equation}\label{eq:2}
    \text{Dado cualquier número racional } y>0, \exists n\in\mathbb{N} \text{ que satisfaga } \frac{1}{n}<y  \text{ o }1<ny
\end{equation}

Para (\ref{eq:1}), supóngase $x=\frac{n}{m}$, entonces existen dos casos: \begin{enumerate}
    \item $(x<0)$, se cumple la propiedad (\ref{eq:1}) en todos los casos por definición.
    \item $(x\geq 0)$ Como sabemos $x=\frac{n}{m}$ y $m\neq 0$. Entonces, por definición de números racionales, se debe cumplir $m\geq 1$ en cualquier caso.
\end{enumerate}
Para (\ref{eq:2}), si tomamos $y=\frac{n}{m}\implies m\cdot y=n\implies $ i.e. $n=my\geq y$ (como $m\geq 1)\implies n+1>y$, cumpliendo la propiedad (\ref{eq:2}).\newline \newline
$\therefore$ el campo $\mathbb{Q}$ es arquimediano.
\end{proof}
\begin{enumerate}
    \item \textbf{¿Depende este resultado del orden definido en $\mathbb{Q}$?} \newline \newline 
    En la argumentación de la prueba se asumió que $\mathbb{Q}$ era un campo ordenado. Por lo cual, se determinó que $\mathbb{Q}$ es un campo arquimediano. Sin embargo, no depende del orden definido, ya que solamente existe una relación que ordena a $\mathbb{Q}$ y que únicamente en esa relación puede ser ordenada; como se observa en la demostración anterior.
 \end{enumerate}


2. \textbf{Presente un ejemplo de un campo ordenado no arquimediano.}\newline \newline 
Algunos ejemplos de campos no arquimedianos pero sí ordenados son:  campo de Levi-Civita , los números hiperreales, números surreales, el campo de Dehn, y el campo de las funciones racionales. De los cuales, se consideró el campo de las \textit{funciones racionales}.\newline \newline \textbf{Funciones racionales}\newline 

Por definición de un campo ordenado, supóngase dos funciones $f(x),g(x)\in\mathbb{Q}(x)$ y asumimos que $f<g\leftrightarrow f\neq g$ y que al substraer $g-f$ el resultado es un coeficiente positivo. Entonces, por las propiedades del orden, se deben cumplir: 
\begin{enumerate}
    \item Según la tricotomía, por cada $f\in \mathbb{Q}(x)$, se tiene que mantener $f>0,f=0$ o $0>f$
    \item Si hay tres elementos $f,g,h\in \mathbb{Q}(x)$ y $f<g$, entonces $f+h<g+h$
    \item Si hay tres elementos $f,g,h\in \mathbb{Q}(x),f<g$ y $0<h$, entonces $fh<gh$. 
\end{enumerate}

\begin{proof}
Para mostrar que las funciones racionales son ordenadas, asúmase que $0<f\leftrightarrow -f<0$, entonces cualquier elemento de $\mathbb{Q}(x)$ puede escribir como $\frac{f(x)}{g(x)}$ donde $g>0$. Por otra parte, si definimos $\frac{f_1(x)}{g_1(x)},\frac{f_2(x)}{g_2(x)}\in\mathbb{Q}(x)$, en donde por definición $g_1>0$ y $g_2 > 0$, eso quiere decir que: 
$$\implies \frac{f_1(x)}{g_1(x)}<\frac{f_2(x)}{g_2(x)}\implies f_1(x)g_2(x)<f_2(x)g_1(x)$$
probando que las funciones racionales son ordenadas.
\end{proof}

Por otra parte queremos determinar que las funciones racionales no son arquimedeanas: 

\begin{proof}
Considérese un contraejemplo, tomando en cuenta las funciones racionales $f(x)=x$ y $g(x) = 1$. Se observa que si se toma un elemento  $n \in \mathbb{N}$, siempre habrá una función $f(x)>n\cdot g(x)$, debido a que $\big(f-n\cdot g\big)(x)=x-n$ y afirmando que el coeficiente de $\big(f-n\cdot g\big)$ es $1$, el cual debe ser positivo; incumpliendo la propiedad arquimediana.
\end{proof}

3. \textbf{Si $a$ y $b$ son elementos de un campo arquimediano $F$ con $a<b$, demuestre que existe un racional $c \in F \ni a<c<b$.}
\begin{proof}
Basándose en la deducción de \cite{bartle2000introduction}. 
Supóngase que c=0, por lo que se puede considerar $a>0 .$ En donde $b-a>0$. Siguiendo el corolario que dice si $t>0,$ entonces existe $n_{t} \in \mathbb{N}$ tal que $0<1 / n_{t}<t$. Se puede asumir que $n \in \mathbb{N}$ tal que $1 / n<b-a .$ Por lo tanto, si se tiene $n a+1<n b .$, aplicando nuevamente otro corolario que afirma: Si $b>0,$ entonces existe $n_{b} \in \mathbb{N}$ tal que $n_{b}-1 \leq y \leq n$. Por lo que se tiene que: $n a>0,$ entonces $m \in \mathbb{N}$ con $m-1 \leq n a<m .$ Por lo tanto, $m \leq n a+1<n b,$ Mientras que $n a<m<n b .$ Entonces el número racional $r:=m / n$ satisface $a<c<b$

\end{proof}

4.\newline
\begin{itemize}
    \item Defina operaciones en $\mathbb{Q}(\sqrt{2})=\{x+(\sqrt{2}) y: x, y \in\mathbb{Q}\}$ de tal forma que sea un campo.
    \begin{proof}
    Por definición, asúmase que $\mathbb{Q}(\sqrt{2})\subset \mathbb{R}$, por lo que las propiedades de asociatividad, conmutatividad  y distributividad de la multiplicación sobre la suma son implícitas por ser un subcampo de $\mathbb{R}$. Entonces: 
    \begin{enumerate}
        \item {Cerradura bajo la adición.} Dados $a_1,a_2\in \mathbb{Q}(\sqrt{2})$, entonces $a_1+a_2 \in \mathbb{Q}(\sqrt{2})$.\newline \newline 
        Sea $a_1= x_1+(\sqrt{2})y_1$ y $a_2= x_2+(\sqrt{2})y_2\in\mathbb{Q}(\sqrt{2}) \implies a_1+a_2 = x_1+x_2 +(\sqrt{2})y_1+(\sqrt{2})y_2=x_1+x_2+(\sqrt{2})[y_1+y_2]\in \mathbb{Q}(\sqrt{2})$.
        
        \item {Cerradura bajo la multiplicación.} Dados $a_1,a_2\in \mathbb{Q}(\sqrt{2})$, entonces $a_1\cdot a_2 \in \mathbb{Q}(\sqrt{2})$.\newline\newline 
         Sea $a_1= x_1+(\sqrt{2})y_1$ y $a_2= x_2+(\sqrt{2})y_2\in\mathbb{Q}(\sqrt{2}) \implies a_1\cdot a_2 = (x_1+(\sqrt{2})y_1)(x_2+(\sqrt{2})y_2)= x_1\cdot x_2 +(\sqrt{2})y_2x_1 +(\sqrt{2})y_1x_2 +(\sqrt{2}y_1)\cdot(\sqrt{2}y_2) = x_1\cdot x_2 +(\sqrt{2})[y_2x_1 +y_1x_2] +2(y_1y_2)\in \mathbb{Q}(\sqrt{2})$
        \item {Inverso.}  Sea $a \in \mathbb{Q}(\sqrt{2})\implies \frac{1}{a} \in \mathbb{Q}(\sqrt{2}) $ \newline \newline 
        $$
\begin{aligned}
(x+y \sqrt{2})^{-1} &=\frac{1}{x+y \sqrt{2}} \\
&=\frac{x}{x^{2}-2 y^{2}}-\frac{y}{x^{2}-2 y^{2}} \sqrt{2} \in \mathbb{Q}(\sqrt{2})
\end{aligned}
$$
en donde $x^{2}-2 y^{2} \neq 0$ porque $\sqrt{2}$ es irracional. $\implies \frac{1}{a}\in \mathbb{Q}(\sqrt{2})$
        
    \end{enumerate}
    
    $\therefore \mathbb{Q}(\sqrt{2})=\{x+(\sqrt{2}) y: x, y \in\mathbb{Q}\}$ es un campo. 
    \end{proof}
    \item ¿Es un campo ordenado?
    \begin{proof}
    Dadas las propiedades de un campo ordenado (Sea $P$ la clase positiva del campo $\mathbb{F}$, entonces se dice que $\mathbb{F}$ está ordenada por $P$ (O que $\mathbb{F}$ es un campo ordenado): 
    \begin{enumerate}
        \item  Si $a \in P$, se dice que a es positivo. Notación $a>0$.
        \item Si $a\in$ P o $a=0,$ se dice que $a$ es no negativo. Notación $a \geq 0$
        \item Si $a, b \in \mathbb{F}$ y $a-b \in P,$, se escribe $a>b$
        \item Si $a, b \in \mathbb{F}$ y $a-b \in P$ o $a-b=0$, se escribe $a\geq b$ 
    \end{enumerate}
     Se propone un $a=x+(\sqrt{2}y)$. Debido a la relación $\mathbb{Q}(\sqrt{2})\subset R$ $\implies a>0 $ o $a\geq 0$, demostrando las propiedades (1) y (2).  Por otra parte, se proponen dos elementos $ x_1+(\sqrt2)x_2$ y $y_1+(\sqrt2)y_2$, tal que:
    $$
    x_1+(\sqrt2)x_2 \leq y_1+(\sqrt2)y_2\iff y_1-x_1+(x_2-y_2)(\sqrt2) \geq 0
    $$
    Demostrando las propiedades (3) y (4). $\therefore \mathbb{Q}(\sqrt{2})=\{x+(\sqrt{2}) y: x, y \in\mathbb{Q}\}$ es un campo ordenado.   
    
    \end{proof}
    \item ¿Es ordenado el campo de los complejos?
    \begin{proof}
    Asúmase un contraejemplo. Dígase un elemento $-i\in\mathbb{C}$. Entonces, se tienen dos casos: (i) Si $-i>0$, es decir que si se eleva al cuadrado $(-i)^2>0\implies -1>0$. Entonces se tiene que $-i<0$. (ii) Si se eleva a una potencia 4 $\implies (-i)^4>0 \implies (-i)^2 (-i)^2 = (-1)(-1) = 1>0$. Afirmando al mismo tiempo que $-1>0$ y $1>0$ $(\to\gets)$. Es decir, una contradicción en una de las propiedades de un campo ordenado (Si $0 < a < b$ y $0 < c < d$, entonces $a c < b d$). Por lo tanto, el campo de los complejos no es ordenado.   
    \end{proof}
\end{itemize}


\bibliographystyle{apalike}
\bibliography{sample.bib}

\end{document}