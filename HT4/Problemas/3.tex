\section{Problema 3.} Pruebe que si la función $f: I \rightarrow \mathbb{R}$ tiene derivada acotada sobre $I$, entonces $f$ es
uniformemente continua sobre $I$.

\begin{proof}
A probar: $f$ es uniformemente continua sobre $I$. Por hipótesis, tenemos que $f$ tiene derivada acotada sobre $I$, que se puede expresar como: $$|f'(x)|\leq M,\qquad \forall \ x\in I.$$

Notamos que se cumplen las condiciones para aplicar el teorema del Valor Medio de \cite{bartle2000introduction}.
\begin{tcolorbox}[colback=gray!15,colframe=gray!1!gray,title=Teorema del Valor Medio 6.2.4 de \cite{bartle2000introduction}.]
Suppose that $f$ is continuous on a closed interval $I:=[a, b]$, and that $f$ has a derivative in the open interval $(a, b) .$ Then there exists at least one point $c$ in $(a, b)$ such that
$$
f(b)-f(a)=f^{\prime}(c)(b-a)
$$
\end{tcolorbox}

Ahora, nos interesa que la función sea Lipschitz para aplicar ciertas propiedades previamente demostradas en clase. Entonces, sean $a,b\in I$, tal que: 
\begin{align*}
    |f(b)-f(a)|&=|f'(c)||b-a|\\
               &= |f'(c)(b-a)|\\
               &\leq K|b-a|.
\end{align*}


\begin{tcolorbox}[colback=blue!15,colframe=blue!1!blue,title=Definición 5.4.4 de \cite{bartle2000introduction}.]
Let $A \subseteq \mathbb{R}$ and let $f: A \rightarrow \mathbb{R}$. If there exists a constant $K>0$ such that
$$
|f(x)-f(u)| \leq K|x-u|
$$
for all $x, u \in A$, then $f$ is said to be a Lipschitz function (or to satisfy a Lipschitz condition) on $A$.
\end{tcolorbox}

Ahora bien, ya que sabemos que es una función Lipschitz, podemos aplicar el teorema 5.4.5 de \cite{bartle2000introduction}.
\begin{tcolorbox}[colback=gray!15,colframe=gray!1!gray,title=Teorema 5.4.5 de \cite{bartle2000introduction}.]
If $f: A \rightarrow \mathbb{R}$ is a Lipschitz function, then $f$ is uniformly continuous on $A$
\end{tcolorbox}
$\therefore \ f$ es uniformemente continua sobre $I$.  

\end{proof}
