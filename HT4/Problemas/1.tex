\section{Problema 1.} Sea $f:(0,1) \rightarrow \mathbb{R}$ una función continua que satisface $[f(x)]^{2}=1, \forall x \in(0,1)$. Pruebe
que $f \equiv 1$ o $f \equiv-1$.
\begin{noter}{Notación}
$f \equiv 1$ hace referencia a:
$$f(x)=1, \qquad \forall \ x\in(0,1).$$
Análogamente, $f \equiv -1$ hace referencia a:
$$f(x)=-1, \qquad \forall \ x\in(0,1).$$
\end{noter}
\begin{proof}
A probar: $f\equiv 1$ o $f\equiv-1$. Sabemos que $f(x)$ es una función continua, en donde, 
$$[f(x)]^2=f(x)f(x).$$
Por el inciso a del teorema 5.2.2 de \cite{rudin1976principles} entonces $[f(x)]^2$ debe ser continua en (0,1) también. 
\begin{tcolorbox}[colback=gray!15,colframe=gray!1!gray,title=Teorema 5.2.2 de \cite{bartle2000introduction}]
Let $A \subseteq \mathbb{R}$, let $f$ and $g$ be continuous on $A$ to $\mathbb{R}$, and let $b \in \mathbb{R}$.
\begin{enumerate}
    \item The functions $f+g, f-g, f g$, and bf are continuous on A.
    \item If $h: A \rightarrow \mathbb{R}$ is continuous on $A$ and $h(x) \neq 0$ for $x \in A$, then the quotient $f / h$ is continuous on $A$.
\end{enumerate}
\end{tcolorbox}
Ahora bien, nótese que:
$$\implies [f(x)]^2=1 \implies f(x)= \pm \sqrt{1} = \pm 1, \qquad \forall \ x\in(0,1).$$
Considérese la definición 6.1.1 de diferenciabilidad de \cite{bartle2000introduction}. 
\begin{tcolorbox}[colback=blue!15,colframe=blue!1!blue,title=Definición 6.1.1. de \cite{bartle2000introduction}]
Let $I \subseteq \mathbb{R}$ be an interval, let $f: I \rightarrow \mathbb{R}$, and let $c \in I$. We say that a real number $L$ is the derivative of $f$ at $c$ if given any $\varepsilon>0$ there exists $\delta(\varepsilon)>0$ such that if $x \in I$ satisfies $0<|x-c|<\delta(\varepsilon)$, then
$$
\left|\frac{f(x)-f(c)}{x-c}-L\right|<\varepsilon
$$
In this case we say that $f$ is differentiable at $c$, and we write $f^{\prime}(c)$ for $L$. In other words, the derivative of $f$ at $c$ is given by the limit
$$
f^{\prime}(c)=\lim _{x \rightarrow c} \frac{f(x)-f(c)}{x-c}
$$
provided this limit exists. (We allow the possibility that $c$ may be the endpoint of the interval.)
\end{tcolorbox}

Sea $c\in (0,1)$. Dado un $\epsilon>0$ tal que exista $\delta(\epsilon)>0$, entonces tenemos 2 casos ($f(x)=\pm 1$): 
\begin{enumerate}
    \item $$\left|\frac{f(x)-f(c)}{x-c}\right|=\left|\frac{1-1}{x-c}\right|=0<\epsilon$$
    \item  $$\left|\frac{f(x)-f(c)}{x-c}\right|=\left|\frac{(-1)-(-1)}{x-c}\right|=0<\epsilon$$
\end{enumerate}
$\implies f$ es diferenciable en $c$ y $f'(c)=0$.  
$\implies$ Ahora tomemos en cuenta el inciso 2 del teorema 5.11 de \cite{rudin1976principles}.
\begin{tcolorbox}[colback=gray!15,colframe=gray!1!gray,title=Teorema 5.11 de \cite{rudin1976principles}]
Suppose $f$ is differentiable in $(a, b)$.
\begin{enumerate}
    \item If $f^{\prime}(x) \geq 0$ for all $x \in(a, b)$, then $f$ is monotonically increasing.
    \item If $f^{\prime}(x)=0$ for all $x \in(a, b)$, then $f$ is constant.
    \item If $f^{\prime}(x) \leq 0$ for all $x \in(a, b)$, then $f$ is monotonically decreasing.
\end{enumerate}
\end{tcolorbox}
$\implies f$ es constante en $(0,1)$. Ahora tenemos que $f(x)=c, \quad \forall \ x\in(0,1)$. Por lo cual, regresamos a la expresión original, en donde:
$$\implies [c]^2=1 \implies c=\pm\sqrt{1}\implies c=\pm 1.$$
$\therefore \ f\equiv 1$ o $f\equiv-1$. 
\end{proof}