\subsection{Capítulo 4: Problema 6}

Suponga $E$ es compacto, pruebe que $f$ es continua en $E$ si y solo si su gráfica es compacta. 
\newline 
\textbf{IMPORTANTE:} El problema proporciona 2 comentarios:
\begin{noter}{Gráfica}
Si $f$ está definido sobre $E$, la gráfica de $f$ es el conjunto de puntos $(x,f(x))$, para $x\in E$.
\end{noter}
\begin{noter}{Caso particular}
En particular, si $E$ es un conjunto de números reales y $f$ se evalúa con reales; la gráfica de $f$ es un subconjunto del plano.
\end{noter}
\begin{noter}{Nota 1}
Para la argumentación de esta prueba, se tomó como referencia la aclaración de \cite{3534792} respecto al producto topológico que es necesario que tenga la \textbf{gráfica} del problema; que aparentemente es una parte importante que \cite{rudin1976principles} asumió sin especificarlo correctamente.

«Por ejemplo, supóngase: 
\begin{enumerate}
    \item Se tiene que $E=Y=[0,1]$ bajo la métrica usual y $f(x)=x$ es el mapeo identidad. $\implies f$ es continuo.  \newline\newline
    
    Ahora bien,
    
    \item Se define 
    $$d((e,y),(e',y')) = \begin{cases} 0, & e = e'\text{ and }y = y'\\1, & \text{otherwise}\end{cases}$$
    
    
    Entonces $d$ es una métrica que induce una topología discreta sobre $G$. Como se sabe que $G$ es infinito, la colección $\big\{\{x\}\mid x \in G\big\}$ es una cubierta infinita de $G$ por conjuntos abiertos sin una subcubierta finita. Es decir, $G$ no es compacto.
\end{enumerate}
Por estas razones, para este caso particular se utiliza una topología adherente al problema, que es el producto topológico definido como  $E\times Y$, el cual $G$ hereda como un subespacio.»

\end{noter}

\begin{tcolorbox}[colback=blue!15,colframe=blue!1!blue,title=Definición de producto topológico de \cite{naber1997topology}]
La definición trata con conceptos avanzados de topología, que de momento parecen muy abstractos, sin embargo es la literatura que fundamenta el producto topológico y reafirma lo de la \textbf{Nota 1}. Capítulo 1.3 - \textit{Products and Locals Products} - Página 56 de \cite{naber1997topology}.
\end{tcolorbox}
\begin{tcolorbox}[colback=gray!15,colframe=gray!1!gray,title= Teorema 4.14 de \cite{rudin1976principles} ]
Supóngase que $f$ es un mapeo continuo de un espacio métrico compacto $X$ en un espacio métrico $Y$. Entonces $f(Y)$ es compacto. 
\end{tcolorbox}

%---------------------------------
\begin{proof}

Primero, vamos a reescribir el problema incluyendo el concepto de gráfica con su topología especificada en la nota 1 y la definición de producto topológico. Es decir, que tenemos un mapeo $f: E\to Y$ para las métricas $E$ y $Y$, en donde $E$ es un compacto.
\begin{noter}{Notación}
Vamos a denotar $G$ como la gráfica y se define como, $$G:=\{\Psi(x)=(x,f(x)) \ | \ x\in E\}$$
\end{noter}
Entonces, nos piden probar,
\begin{center}
   $f$ es continua $\Longleftrightarrow$ $\Psi\subset E\times Y$ es compacto. 
\end{center}
\begin{itemize}
    \item [\fbox{$\to$}] Sabemos que $f$ es continua y que $E$ es compacto. A probar: $\Psi\subset E\times Y$ es un compacto.\newline\newline 
    Ahora bien nos interesa encontrar un mapeo continuo que sea compacto, por lo que asumimos que existen un par de puntos $(x, y), x \in E, y \in Y$, con una métrica definida como: $$\rho\left(\left(x_{1}, y_{1}\right),\left(x_{2}, y_{2}\right)\right)=d_{E}\left(x_{1}, x_{2}\right)+d_{Y}\left(y_{1}, y_{2}\right),$$ basado en la definición usual de \cite{rudin1976principles}. Por las propiedades de la gráfica $\Psi(x)$ debe ser continua si $f$ es continua. Por otro lado, sea $x \in X$ y dado $\varepsilon>0$. Elegimos arbitrariamente un $\eta>0$ tal que $$d_{Y}(f(x), f(u))<\frac{\varepsilon}{2}$$ si $d_{E}(x, y)<\eta .$ Entonces, nuevamente, elegimos arbitrariamente un $\delta=\min \left(\eta, \frac{\varepsilon}{2}\right) .$  Por lo tanto, se puede observar que 
    $$\rho(\Psi(x), \Psi(u))<\varepsilon$$
    si $d_{E}(x, u)<\delta$. 
     Como sabemos que $\Psi(x)$ es continuo, entonces por la siguiente desigualdad
$$\rho(\Psi(x), \Psi(u)) \geq d_{Y}(f(x), f(u))$$ observamos que $f$ debe ser continua. 
Finalmente, basándonos en el teorema 4.14 de \cite{rudin1976principles} se deduce que la gráfica de una función $f$ sobre un conjunto compacto $E$ es compacto. 


    \item [\fbox{$\gets$}] Sabemos que $\Psi\subset E\times Y$ y $E$ son compactos. A probar: $f$ es continua.\newline\newline 
    
    Ahora, por contrapuesta: asumamos que $f$ no es continuo en un punto  $x$. $\implies$ Hay una sucesión de puntos $x_{n}$ que convergen a $x$ tales que $f\left(x_{n}\right)$ no convergen a $f(x)$. $\implies$ Por Bolzano-Weierstrass de \cite{abbott2012understanding}, si ninguna subsucesión de $f\left(x_{n}\right)$ converge, entonces la sucesión $\left\{\left(x_{n}, f\left(x_{n}\right)\right\}_{n=1}^{\infty}\right.$ no tiene ninguna subsucesión convergente, y por lo tanto su gráfica no es compacta. $\implies$ Ahora bien, si una subsucesión de $f\left(x_{n}\right)$ converge, dígase $f\left(x_{n_{k}}\right) \rightarrow z$, pero $z \neq f(x)$. $\implies$ La gráfica de $f$ falla en contener los puntos límite de $(x, z)$, y por lo tanto no es cerrado. Por el teorema de Heine-Borel de \cite{abbott2012understanding}, entonces no es compacta. 
    
\end{itemize}
\end{proof}